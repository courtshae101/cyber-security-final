% Options for packages loaded elsewhere
\PassOptionsToPackage{unicode}{hyperref}
\PassOptionsToPackage{hyphens}{url}
%
\documentclass[
]{article}
\usepackage{lmodern}
\usepackage{amsmath}
\usepackage{ifxetex,ifluatex}
\ifnum 0\ifxetex 1\fi\ifluatex 1\fi=0 % if pdftex
  \usepackage[T1]{fontenc}
  \usepackage[utf8]{inputenc}
  \usepackage{textcomp} % provide euro and other symbols
  \usepackage{amssymb}
\else % if luatex or xetex
  \usepackage{unicode-math}
  \defaultfontfeatures{Scale=MatchLowercase}
  \defaultfontfeatures[\rmfamily]{Ligatures=TeX,Scale=1}
\fi
% Use upquote if available, for straight quotes in verbatim environments
\IfFileExists{upquote.sty}{\usepackage{upquote}}{}
\IfFileExists{microtype.sty}{% use microtype if available
  \usepackage[]{microtype}
  \UseMicrotypeSet[protrusion]{basicmath} % disable protrusion for tt fonts
}{}
\makeatletter
\@ifundefined{KOMAClassName}{% if non-KOMA class
  \IfFileExists{parskip.sty}{%
    \usepackage{parskip}
  }{% else
    \setlength{\parindent}{0pt}
    \setlength{\parskip}{6pt plus 2pt minus 1pt}}
}{% if KOMA class
  \KOMAoptions{parskip=half}}
\makeatother
\usepackage{xcolor}
\IfFileExists{xurl.sty}{\usepackage{xurl}}{} % add URL line breaks if available
\IfFileExists{bookmark.sty}{\usepackage{bookmark}}{\usepackage{hyperref}}
\hypersetup{
  pdftitle={State\_and\_indiv},
  pdfauthor={Courtney},
  hidelinks,
  pdfcreator={LaTeX via pandoc}}
\urlstyle{same} % disable monospaced font for URLs
\usepackage[margin=1in]{geometry}
\usepackage{color}
\usepackage{fancyvrb}
\newcommand{\VerbBar}{|}
\newcommand{\VERB}{\Verb[commandchars=\\\{\}]}
\DefineVerbatimEnvironment{Highlighting}{Verbatim}{commandchars=\\\{\}}
% Add ',fontsize=\small' for more characters per line
\usepackage{framed}
\definecolor{shadecolor}{RGB}{248,248,248}
\newenvironment{Shaded}{\begin{snugshade}}{\end{snugshade}}
\newcommand{\AlertTok}[1]{\textcolor[rgb]{0.94,0.16,0.16}{#1}}
\newcommand{\AnnotationTok}[1]{\textcolor[rgb]{0.56,0.35,0.01}{\textbf{\textit{#1}}}}
\newcommand{\AttributeTok}[1]{\textcolor[rgb]{0.77,0.63,0.00}{#1}}
\newcommand{\BaseNTok}[1]{\textcolor[rgb]{0.00,0.00,0.81}{#1}}
\newcommand{\BuiltInTok}[1]{#1}
\newcommand{\CharTok}[1]{\textcolor[rgb]{0.31,0.60,0.02}{#1}}
\newcommand{\CommentTok}[1]{\textcolor[rgb]{0.56,0.35,0.01}{\textit{#1}}}
\newcommand{\CommentVarTok}[1]{\textcolor[rgb]{0.56,0.35,0.01}{\textbf{\textit{#1}}}}
\newcommand{\ConstantTok}[1]{\textcolor[rgb]{0.00,0.00,0.00}{#1}}
\newcommand{\ControlFlowTok}[1]{\textcolor[rgb]{0.13,0.29,0.53}{\textbf{#1}}}
\newcommand{\DataTypeTok}[1]{\textcolor[rgb]{0.13,0.29,0.53}{#1}}
\newcommand{\DecValTok}[1]{\textcolor[rgb]{0.00,0.00,0.81}{#1}}
\newcommand{\DocumentationTok}[1]{\textcolor[rgb]{0.56,0.35,0.01}{\textbf{\textit{#1}}}}
\newcommand{\ErrorTok}[1]{\textcolor[rgb]{0.64,0.00,0.00}{\textbf{#1}}}
\newcommand{\ExtensionTok}[1]{#1}
\newcommand{\FloatTok}[1]{\textcolor[rgb]{0.00,0.00,0.81}{#1}}
\newcommand{\FunctionTok}[1]{\textcolor[rgb]{0.00,0.00,0.00}{#1}}
\newcommand{\ImportTok}[1]{#1}
\newcommand{\InformationTok}[1]{\textcolor[rgb]{0.56,0.35,0.01}{\textbf{\textit{#1}}}}
\newcommand{\KeywordTok}[1]{\textcolor[rgb]{0.13,0.29,0.53}{\textbf{#1}}}
\newcommand{\NormalTok}[1]{#1}
\newcommand{\OperatorTok}[1]{\textcolor[rgb]{0.81,0.36,0.00}{\textbf{#1}}}
\newcommand{\OtherTok}[1]{\textcolor[rgb]{0.56,0.35,0.01}{#1}}
\newcommand{\PreprocessorTok}[1]{\textcolor[rgb]{0.56,0.35,0.01}{\textit{#1}}}
\newcommand{\RegionMarkerTok}[1]{#1}
\newcommand{\SpecialCharTok}[1]{\textcolor[rgb]{0.00,0.00,0.00}{#1}}
\newcommand{\SpecialStringTok}[1]{\textcolor[rgb]{0.31,0.60,0.02}{#1}}
\newcommand{\StringTok}[1]{\textcolor[rgb]{0.31,0.60,0.02}{#1}}
\newcommand{\VariableTok}[1]{\textcolor[rgb]{0.00,0.00,0.00}{#1}}
\newcommand{\VerbatimStringTok}[1]{\textcolor[rgb]{0.31,0.60,0.02}{#1}}
\newcommand{\WarningTok}[1]{\textcolor[rgb]{0.56,0.35,0.01}{\textbf{\textit{#1}}}}
\usepackage{graphicx}
\makeatletter
\def\maxwidth{\ifdim\Gin@nat@width>\linewidth\linewidth\else\Gin@nat@width\fi}
\def\maxheight{\ifdim\Gin@nat@height>\textheight\textheight\else\Gin@nat@height\fi}
\makeatother
% Scale images if necessary, so that they will not overflow the page
% margins by default, and it is still possible to overwrite the defaults
% using explicit options in \includegraphics[width, height, ...]{}
\setkeys{Gin}{width=\maxwidth,height=\maxheight,keepaspectratio}
% Set default figure placement to htbp
\makeatletter
\def\fps@figure{htbp}
\makeatother
\setlength{\emergencystretch}{3em} % prevent overfull lines
\providecommand{\tightlist}{%
  \setlength{\itemsep}{0pt}\setlength{\parskip}{0pt}}
\setcounter{secnumdepth}{-\maxdimen} % remove section numbering
\ifluatex
  \usepackage{selnolig}  % disable illegal ligatures
\fi

\title{State\_and\_indiv}
\author{Courtney}
\date{4/5/2021}

\begin{document}
\maketitle

\begin{Shaded}
\begin{Highlighting}[]
\FunctionTok{library}\NormalTok{(tidyverse)}
\end{Highlighting}
\end{Shaded}

\begin{verbatim}
## -- Attaching packages --------------------------------------- tidyverse 1.3.0 --
\end{verbatim}

\begin{verbatim}
## v ggplot2 3.3.3     v purrr   0.3.4
## v tibble  3.0.6     v dplyr   1.0.4
## v tidyr   1.1.2     v stringr 1.4.0
## v readr   1.4.0     v forcats 0.5.1
\end{verbatim}

\begin{verbatim}
## -- Conflicts ------------------------------------------ tidyverse_conflicts() --
## x dplyr::filter() masks stats::filter()
## x dplyr::lag()    masks stats::lag()
\end{verbatim}

\begin{Shaded}
\begin{Highlighting}[]
\FunctionTok{library}\NormalTok{(ggpubr)}
\end{Highlighting}
\end{Shaded}

\begin{verbatim}
## Warning: package 'ggpubr' was built under R version 4.0.4
\end{verbatim}

\begin{Shaded}
\begin{Highlighting}[]
\NormalTok{breaches }\OtherTok{\textless{}{-}} \FunctionTok{read\_csv}\NormalTok{(}\StringTok{\textquotesingle{}Cyber Security Breaches (1).csv\textquotesingle{}}\NormalTok{,}
                     \AttributeTok{col\_types =} \FunctionTok{cols}\NormalTok{(}
                      \AttributeTok{State =} \FunctionTok{col\_factor}\NormalTok{(),}
                      \AttributeTok{Individuals\_Affected =} \FunctionTok{col\_integer}\NormalTok{()}
\NormalTok{                     )}
\NormalTok{                  )}
\end{Highlighting}
\end{Shaded}

\begin{verbatim}
## Warning: Missing column names filled in: 'X1' [1]
\end{verbatim}

\begin{Shaded}
\begin{Highlighting}[]
\FunctionTok{head}\NormalTok{(breaches)}
\end{Highlighting}
\end{Shaded}

\begin{verbatim}
## # A tibble: 6 x 14
##      X1 Number Name_of_Covered_Entity  State Business_Associat~ Individuals_Aff~
##   <dbl>  <dbl> <chr>                   <fct> <chr>                         <int>
## 1     1      0 Brooke Army Medical Ce~ TX    <NA>                           1000
## 2     2      1 Mid America Kidney Sto~ MO    <NA>                           1000
## 3     3      2 Alaska Department of H~ AK    <NA>                            501
## 4     4      3 Health Services for Ch~ DC    <NA>                           3800
## 5     5      4 L. Douglas Carlson, M.~ CA    <NA>                           5257
## 6     6      5 David I. Cohen, MD      CA    <NA>                            857
## # ... with 8 more variables: Date_of_Breach <chr>, Type_of_Breach <chr>,
## #   Location_of_Breached_Information <chr>, Date_Posted_or_Updated <date>,
## #   Summary <chr>, breach_start <date>, breach_end <date>, year <dbl>
\end{verbatim}

\hypertarget{state-variable}{%
\section{State Variable}\label{state-variable}}

\hypertarget{visualizing-distribution-of-state-variable}{%
\subsection{Visualizing Distribution of State
Variable}\label{visualizing-distribution-of-state-variable}}

\begin{Shaded}
\begin{Highlighting}[]
\NormalTok{state\_bar }\OtherTok{\textless{}{-}}\NormalTok{ breaches }\SpecialCharTok{\%\textgreater{}\%}
  \FunctionTok{mutate}\NormalTok{(}\AttributeTok{State =}\NormalTok{ State }\SpecialCharTok{\%\textgreater{}\%} \FunctionTok{fct\_infreq}\NormalTok{() }\SpecialCharTok{\%\textgreater{}\%} \FunctionTok{fct\_rev}\NormalTok{()) }\SpecialCharTok{\%\textgreater{}\%}
  \FunctionTok{ggplot}\NormalTok{(}\FunctionTok{aes}\NormalTok{(}\AttributeTok{x=}\NormalTok{State)) }\SpecialCharTok{+}
  \FunctionTok{geom\_bar}\NormalTok{()}\SpecialCharTok{+}
  \FunctionTok{coord\_flip}\NormalTok{()}

\NormalTok{state\_bar}
\end{Highlighting}
\end{Shaded}

\begin{center}\includegraphics{CourtneyKennedy_files/figure-latex/unnamed-chunk-4-1} \end{center}

\begin{Shaded}
\begin{Highlighting}[]
\NormalTok{count\_state }\OtherTok{\textless{}{-}}\NormalTok{ breaches }\SpecialCharTok{\%\textgreater{}\%}
  \FunctionTok{mutate}\NormalTok{(}\AttributeTok{State =}\NormalTok{ State }\SpecialCharTok{\%\textgreater{}\%} \FunctionTok{fct\_infreq}\NormalTok{() }\SpecialCharTok{\%\textgreater{}\%} \FunctionTok{fct\_rev}\NormalTok{()) }\SpecialCharTok{\%\textgreater{}\%}
  \FunctionTok{count}\NormalTok{(State)}

\NormalTok{count\_state}
\end{Highlighting}
\end{Shaded}

\begin{verbatim}
## # A tibble: 52 x 2
##    State     n
##  * <fct> <int>
##  1 VT        1
##  2 ME        1
##  3 SD        1
##  4 HI        1
##  5 ID        2
##  6 ND        3
##  7 DE        3
##  8 MT        4
##  9 NH        4
## 10 WY        4
## # ... with 42 more rows
\end{verbatim}

\textbf{- Which values are the most common? Why?}

Breaches in the State of California are the most common since they have
the most breaches at 113. This is most likely due to the fact that
California is highly populated with lots of buisnesses and tech
industries, therefore can have more opportunities for breaches.

\textbf{- Which values are rare? Why? Does that match your
expectations?} The most rare values are VT, ME, SD, and HI which all
have only one breach. Since these are not very largely populated states
this does make sense.

\textbf{- Can you see any unusual patterns? What might explain them?}

There does not appear to be any unusal patterns in the State breach
count. Some states have more breaches than others but there is not any
outliers of cycles of number of breaches.

\textbf{- Are there clusters in the data? If so,} No there are no
clusters in the data, all of the data is relatively evenly distributed.

\textbf{- How are the observations within each cluster similar to or
different from each other?}

As mentioned above there are no clusters present.

\textbf{- How can you explain or describe the clusters?}

As mentioned above there are no clusters present.

\hypertarget{unusual-values}{%
\paragraph{3.1.2.2 Unusual values}\label{unusual-values}}

\textbf{- Describe and demonstrate how you determine if there are
unusual values in the data. E.g. too large, too small, negative, etc.}

\begin{Shaded}
\begin{Highlighting}[]
\NormalTok{breaches }\SpecialCharTok{\%\textgreater{}\%}
  \FunctionTok{mutate}\NormalTok{(}\AttributeTok{State =}\NormalTok{ State }\SpecialCharTok{\%\textgreater{}\%} \FunctionTok{fct\_infreq}\NormalTok{() }\SpecialCharTok{\%\textgreater{}\%} \FunctionTok{fct\_rev}\NormalTok{()) }\SpecialCharTok{\%\textgreater{}\%}
  \FunctionTok{ggplot}\NormalTok{(}\FunctionTok{aes}\NormalTok{(}\AttributeTok{x=}\NormalTok{State)) }\SpecialCharTok{+}
  \FunctionTok{geom\_bar}\NormalTok{()}\SpecialCharTok{+}
  \FunctionTok{coord\_flip}\NormalTok{()}
\end{Highlighting}
\end{Shaded}

\begin{center}\includegraphics{CourtneyKennedy_files/figure-latex/unnamed-chunk-5-1} \end{center}

There were no negative state breaches and no values that were
unexpectadly high or low. This is seen in the bar graph. More
exploration has to be done to determine if any values should be removed.

\textbf{- Describe and demonstrate how you determine if they are
outliers.}

An outlier is 1.5 times the interquartile range away from either the
lower or upper quartile. In order to determine if any of the state count
values are outliers the interquartile range, first quartile, and third
quartile need to be calculated. The State count data then has to be
filtered for values that are less than the first quartile minus the IQR
times 1.5 and values that are greater than the third quartile plus the
IQR times 1.5. The outliers can be seen in the outlier list data frame,
it includes, TX, CA, FL.

\begin{Shaded}
\begin{Highlighting}[]
\NormalTok{state\_count }\OtherTok{\textless{}{-}}\NormalTok{ breaches }\SpecialCharTok{\%\textgreater{}\%}
  \FunctionTok{group\_by}\NormalTok{(State) }\SpecialCharTok{\%\textgreater{}\%}
  \FunctionTok{count}\NormalTok{()}
\NormalTok{state\_count}
\end{Highlighting}
\end{Shaded}

\begin{verbatim}
## # A tibble: 52 x 2
## # Groups:   State [52]
##    State     n
##    <fct> <int>
##  1 TX       83
##  2 MO       25
##  3 AK        5
##  4 DC        9
##  5 CA      113
##  6 PA       40
##  7 TN       32
##  8 NY       58
##  9 NC       32
## 10 MI       26
## # ... with 42 more rows
\end{verbatim}

\begin{Shaded}
\begin{Highlighting}[]
\NormalTok{stdev }\OtherTok{\textless{}{-}}  \FunctionTok{sd}\NormalTok{(state\_count}\SpecialCharTok{$}\NormalTok{n, }\AttributeTok{na.rm =} \ConstantTok{TRUE}\NormalTok{)}
\NormalTok{stdev}
\end{Highlighting}
\end{Shaded}

\begin{verbatim}
## [1] 21.85544
\end{verbatim}

\begin{Shaded}
\begin{Highlighting}[]
\NormalTok{innerQ }\OtherTok{\textless{}{-}}  \FunctionTok{IQR}\NormalTok{(state\_count}\SpecialCharTok{$}\NormalTok{n, }\AttributeTok{na.rm =} \ConstantTok{TRUE}\NormalTok{)}
\NormalTok{innerQ}
\end{Highlighting}
\end{Shaded}

\begin{verbatim}
## [1] 22
\end{verbatim}

\begin{Shaded}
\begin{Highlighting}[]
\NormalTok{firstQ }\OtherTok{\textless{}{-}} \FunctionTok{quantile}\NormalTok{(state\_count}\SpecialCharTok{$}\NormalTok{n, }\FloatTok{0.25}\NormalTok{, }\AttributeTok{na.rm =} \ConstantTok{TRUE}\NormalTok{)}
\NormalTok{firstQ }\OtherTok{\textless{}{-}}\NormalTok{ firstQ[[}\DecValTok{1}\NormalTok{]]}

\NormalTok{thirdQ }\OtherTok{\textless{}{-}} \FunctionTok{quantile}\NormalTok{(state\_count}\SpecialCharTok{$}\NormalTok{n, }\FloatTok{0.75}\NormalTok{, }\AttributeTok{na.rm =} \ConstantTok{TRUE}\NormalTok{)}
\NormalTok{thirdQ }\OtherTok{\textless{}{-}}\NormalTok{ thirdQ[[}\DecValTok{1}\NormalTok{]]}

\NormalTok{outlier\_list }\OtherTok{\textless{}{-}}\NormalTok{ state\_count }\SpecialCharTok{\%\textgreater{}\%}
  \FunctionTok{filter}\NormalTok{(n }\SpecialCharTok{\textless{}}\NormalTok{ (firstQ }\SpecialCharTok{{-}}\NormalTok{ innerQ }\SpecialCharTok{*} \FloatTok{1.5}\NormalTok{) }\SpecialCharTok{|} 
\NormalTok{        n }\SpecialCharTok{\textgreater{}}\NormalTok{ (thirdQ }\SpecialCharTok{+}\NormalTok{ innerQ }\SpecialCharTok{*} \FloatTok{1.5}\NormalTok{))}

\NormalTok{outlier\_list}
\end{Highlighting}
\end{Shaded}

\begin{verbatim}
## # A tibble: 3 x 2
## # Groups:   State [3]
##   State     n
##   <fct> <int>
## 1 TX       83
## 2 CA      113
## 3 FL       66
\end{verbatim}

\textbf{- Show how do your distributions look like with and without the
unusual values.}

With the outliers removed the distribution is made narrower with less
variation. Since the largest state breach counts are removed overall the
distribution becomes more similar throughout.

\begin{Shaded}
\begin{Highlighting}[]
\NormalTok{outlier\_state }\OtherTok{=} \FunctionTok{c}\NormalTok{(}\StringTok{"TX"}\NormalTok{, }\StringTok{"CA"}\NormalTok{, }\StringTok{"FL"}\NormalTok{)}

\NormalTok{no\_out\_bar }\OtherTok{\textless{}{-}}\NormalTok{ breaches }\SpecialCharTok{\%\textgreater{}\%}
  \FunctionTok{mutate}\NormalTok{(}\AttributeTok{State =}\NormalTok{ State }\SpecialCharTok{\%\textgreater{}\%} \FunctionTok{fct\_infreq}\NormalTok{() }\SpecialCharTok{\%\textgreater{}\%} \FunctionTok{fct\_rev}\NormalTok{()) }\SpecialCharTok{\%\textgreater{}\%}
  \FunctionTok{filter}\NormalTok{(}\SpecialCharTok{!}\NormalTok{(State }\SpecialCharTok{\%in\%}\NormalTok{ outlier\_state)) }\SpecialCharTok{\%\textgreater{}\%}
  \FunctionTok{ggplot}\NormalTok{(}\FunctionTok{aes}\NormalTok{(}\AttributeTok{x=}\NormalTok{State)) }\SpecialCharTok{+}
  \FunctionTok{geom\_bar}\NormalTok{()}\SpecialCharTok{+}
  \FunctionTok{coord\_flip}\NormalTok{()}\SpecialCharTok{+}
  \FunctionTok{ylim}\NormalTok{(}\DecValTok{0}\NormalTok{, }\DecValTok{110}\NormalTok{) }\SpecialCharTok{+}
  \FunctionTok{labs}\NormalTok{(}\AttributeTok{title =} \StringTok{"Outliers removed"}\NormalTok{)}

\NormalTok{out\_in\_bar }\OtherTok{\textless{}{-}}\NormalTok{ breaches }\SpecialCharTok{\%\textgreater{}\%}
  \FunctionTok{mutate}\NormalTok{(}\AttributeTok{State =}\NormalTok{ State }\SpecialCharTok{\%\textgreater{}\%} \FunctionTok{fct\_infreq}\NormalTok{() }\SpecialCharTok{\%\textgreater{}\%} \FunctionTok{fct\_rev}\NormalTok{()) }\SpecialCharTok{\%\textgreater{}\%}
  \FunctionTok{ggplot}\NormalTok{(}\FunctionTok{aes}\NormalTok{(}\AttributeTok{x=}\NormalTok{State)) }\SpecialCharTok{+}
  \FunctionTok{geom\_bar}\NormalTok{()}\SpecialCharTok{+}
  \FunctionTok{coord\_flip}\NormalTok{()}\SpecialCharTok{+}
  \FunctionTok{ylim}\NormalTok{(}\DecValTok{0}\NormalTok{,}\DecValTok{110}\NormalTok{) }\SpecialCharTok{+}
  \FunctionTok{labs}\NormalTok{(}\AttributeTok{title =} \StringTok{"Outliers included"}\NormalTok{)}

\FunctionTok{ggarrange}\NormalTok{(no\_out\_bar, out\_in\_bar, }\AttributeTok{ncol =} \DecValTok{2}\NormalTok{)}
\end{Highlighting}
\end{Shaded}

\begin{verbatim}
## Warning: Removed 1 rows containing missing values (geom_bar).
\end{verbatim}

\begin{center}\includegraphics{CourtneyKennedy_files/figure-latex/unnamed-chunk-7-1} \end{center}

\textbf{- Discuss whether or not you need to remove unusual values and
why.}

Since the largest values will provide d the most insight into why
breaches are happeing at such a large rate in certain states they should
not be removed.

\hypertarget{missing-values}{%
\paragraph{3.1.2.3 Missing values}\label{missing-values}}

\textbf{- Does this variable include missing values? Demonstrate how you
determine that.}

There are no missing values in the State variable. The method is.na with
the column name can be used and then the vector returned can be turned
into a data frame that represents the number of NA values (TRUE) and non
NA values (FALSE). It can also be confirmed by calling summary() on the
State, which also shows that there are no NA values in the State
variable. There should also be information for all 50 states plus PR and
DC, which is confirmed using unique() to show there are 52 unqique State
values.

\begin{Shaded}
\begin{Highlighting}[]
\NormalTok{missing }\OtherTok{\textless{}{-}} \FunctionTok{is.na}\NormalTok{(breaches}\SpecialCharTok{$}\NormalTok{State)}

\NormalTok{num\_missing }\OtherTok{\textless{}{-}} \FunctionTok{as.data.frame}\NormalTok{(}\FunctionTok{table}\NormalTok{(missing))}

\NormalTok{num\_missing}
\end{Highlighting}
\end{Shaded}

\begin{verbatim}
##   missing Freq
## 1   FALSE 1055
\end{verbatim}

\begin{Shaded}
\begin{Highlighting}[]
\FunctionTok{summary}\NormalTok{(breaches}\SpecialCharTok{$}\NormalTok{State)}
\end{Highlighting}
\end{Shaded}

\begin{verbatim}
##  TX  MO  AK  DC  CA  PA  TN  NY  NC  MI  MA  IL  UT  NV  AZ  RI  PR  FL  NM  CO 
##  83  25   5   9 113  40  32  58  32  26  32  49   9   5  21   7  31  66  10  18 
##  WY  WI  WA  CT  AL  NE  SC  KY  MN  VA  OH  KS  GA  MD  IN  ID  OR  NJ  DE  IA 
##   4  14  25  17  12   9  13  26  21  18  33   7  30  18  40   2  15  20   3   8 
##  OK  AR  MS  LA  NH  MT  WV  ND  HI  SD  ME  VT 
##   6  11   5   7   4   4   5   3   1   1   1   1
\end{verbatim}

\begin{Shaded}
\begin{Highlighting}[]
\NormalTok{breaches}\SpecialCharTok{$}\NormalTok{State }\SpecialCharTok{\%\textgreater{}\%}
  \FunctionTok{unique}\NormalTok{()}
\end{Highlighting}
\end{Shaded}

\begin{verbatim}
##  [1] TX MO AK DC CA PA TN NY NC MI MA IL UT NV AZ RI PR FL NM CO WY WI WA CT AL
## [26] NE SC KY MN VA OH KS GA MD IN ID OR NJ DE IA OK AR MS LA NH MT WV ND HI SD
## [51] ME VT
## 52 Levels: TX MO AK DC CA PA TN NY NC MI MA IL UT NV AZ RI PR FL NM CO ... VT
\end{verbatim}

\textbf{- Demonstrate and discuss how you handle the missing values.
E.g., removing, replacing with a constant value, or a value based on the
distribution, etc.}

There are no missing values so they do no need to be handled.

\textbf{- Show how your data looks in each case after handling missing
values.Describe and discuss the distribution.}

Since there is no missing values the distribution does not changed, see
earlier bar graph for distribution.

\hypertarget{does-converting-the-type-of-this-variable-help-exploring-the-distribution-of-its-values-or-identifying-outliers-or-missing-values-3}{%
\paragraph{3.1.2.4 Does converting the type of this variable help
exploring the distribution of its values or identifying outliers or
missing values?
(3)}\label{does-converting-the-type-of-this-variable-help-exploring-the-distribution-of-its-values-or-identifying-outliers-or-missing-values-3}}

Yes converting State to a logical may be helpful in exploring the
distribution of its values or identifying outliers or missing values
since logical are simpler to evaluate when larger continuous data is
converted into two groups.

\textbf{- What type can the variable be converted to?}

State is of type factor, but it can converted to a logical. By making
the value of State TRUE when the State is in the northeast and FALSE
when the value of the State is not in the northeast, we can see if the
northeast has a large number of breaches. Converting State to a logical
is a simpler way to interpret State values. The converted State type is
saved as a new variable northeast.

\begin{Shaded}
\begin{Highlighting}[]
\NormalTok{northeast\_list }\OtherTok{\textless{}{-}} \FunctionTok{c}\NormalTok{(}\StringTok{"CT"}\NormalTok{, }\StringTok{"MA"}\NormalTok{, }\StringTok{"NH"}\NormalTok{, }\StringTok{"NJ"}\NormalTok{, }\StringTok{"NY"}\NormalTok{, }\StringTok{"PA"}\NormalTok{, }\StringTok{"RI"}\NormalTok{, }\StringTok{"VT"}\NormalTok{, }\StringTok{"DE"}\NormalTok{, }\StringTok{"MD"}\NormalTok{, }\StringTok{"ME"}\NormalTok{)}
\CommentTok{\#function to determine if the states are in the northeast}
\NormalTok{northeast\_check }\OtherTok{\textless{}{-}} \ControlFlowTok{function}\NormalTok{(x) \{}
  \ControlFlowTok{if}\NormalTok{(}\FunctionTok{is.na}\NormalTok{(x))\{}
    \FunctionTok{return}\NormalTok{(}\ConstantTok{NA}\NormalTok{)}
\NormalTok{  \}}
  \ControlFlowTok{else} \ControlFlowTok{if}\NormalTok{(x }\SpecialCharTok{\%in\%}\NormalTok{ northeast\_list)\{}
    \FunctionTok{return}\NormalTok{(}\ConstantTok{TRUE}\NormalTok{)}
\NormalTok{  \}}
  \ControlFlowTok{else}\NormalTok{\{}
    \FunctionTok{return}\NormalTok{(}\ConstantTok{FALSE}\NormalTok{)}
\NormalTok{  \}}
\NormalTok{\}}

\NormalTok{breaches}\SpecialCharTok{$}\NormalTok{northeast }\OtherTok{\textless{}{-}} \FunctionTok{sapply}\NormalTok{(breaches}\SpecialCharTok{$}\NormalTok{State, northeast\_check)}
\FunctionTok{head}\NormalTok{(breaches)}
\end{Highlighting}
\end{Shaded}

\begin{verbatim}
## # A tibble: 6 x 15
##      X1 Number Name_of_Covered_Entity  State Business_Associat~ Individuals_Aff~
##   <dbl>  <dbl> <chr>                   <fct> <chr>                         <int>
## 1     1      0 Brooke Army Medical Ce~ TX    <NA>                           1000
## 2     2      1 Mid America Kidney Sto~ MO    <NA>                           1000
## 3     3      2 Alaska Department of H~ AK    <NA>                            501
## 4     4      3 Health Services for Ch~ DC    <NA>                           3800
## 5     5      4 L. Douglas Carlson, M.~ CA    <NA>                           5257
## 6     6      5 David I. Cohen, MD      CA    <NA>                            857
## # ... with 9 more variables: Date_of_Breach <chr>, Type_of_Breach <chr>,
## #   Location_of_Breached_Information <chr>, Date_Posted_or_Updated <date>,
## #   Summary <chr>, breach_start <date>, breach_end <date>, year <dbl>,
## #   northeast <lgl>
\end{verbatim}

\textbf{- How will the distribution look? Please demonstrate with
appropriate plots.}

From plotting the converted logical State as a bar graph, we can see
that the majority of the breaches were not in the northeast. However the
number of breaches is large for the northeast since there is only 9
states vs the other 43 States and territories. We can also see that
there are no NA values, which confirms the analysis done earlier.

\begin{Shaded}
\begin{Highlighting}[]
\NormalTok{breaches }\SpecialCharTok{\%\textgreater{}\%}
  \FunctionTok{ggplot}\NormalTok{(}\FunctionTok{aes}\NormalTok{(}\AttributeTok{x=}\NormalTok{northeast, }\AttributeTok{fill =}\NormalTok{ northeast)) }\SpecialCharTok{+}
  \FunctionTok{geom\_bar}\NormalTok{()}
\end{Highlighting}
\end{Shaded}

\includegraphics{CourtneyKennedy_files/figure-latex/unnamed-chunk-10-1.pdf}

\hypertarget{what-new-variables-do-you-need-to-create-3}{%
\paragraph{3.1.2.5 What new variables do you need to create?
(3)}\label{what-new-variables-do-you-need-to-create-3}}

\textbf{- List the variables} northeast, westcoast, midwest, south.

All are logical variables that are true or false if the breach is in the
region.

Region, which is a factor variable that sorts the US into northeast,
westcoast, midwest, south and other.

\begin{Shaded}
\begin{Highlighting}[]
\NormalTok{westcoast\_list }\OtherTok{\textless{}{-}} \FunctionTok{c}\NormalTok{(}\StringTok{"WY"}\NormalTok{, }\StringTok{"CO"}\NormalTok{, }\StringTok{"UT"}\NormalTok{, }\StringTok{"NV"}\NormalTok{, }\StringTok{"ID"}\NormalTok{, }\StringTok{"CA"}\NormalTok{, }\StringTok{"OR"}\NormalTok{, }\StringTok{"WA"}\NormalTok{, }\StringTok{"AK"}\NormalTok{, }\StringTok{"AZ"}\NormalTok{, }\StringTok{"NM"}\NormalTok{)}
\CommentTok{\#function to determine if the states are on the West Coast}
\NormalTok{westcoast\_check }\OtherTok{\textless{}{-}} \ControlFlowTok{function}\NormalTok{(x) \{}
  \ControlFlowTok{if}\NormalTok{(}\FunctionTok{is.na}\NormalTok{(x))\{}
    \FunctionTok{return}\NormalTok{(}\ConstantTok{NA}\NormalTok{)}
\NormalTok{  \}}
  \ControlFlowTok{else} \ControlFlowTok{if}\NormalTok{(x }\SpecialCharTok{\%in\%}\NormalTok{ westcoast\_list)\{}
    \FunctionTok{return}\NormalTok{(}\ConstantTok{TRUE}\NormalTok{)}
\NormalTok{  \}}
  \ControlFlowTok{else}\NormalTok{\{}
    \FunctionTok{return}\NormalTok{(}\ConstantTok{FALSE}\NormalTok{)}
\NormalTok{  \}}
\NormalTok{\}}

\NormalTok{breaches}\SpecialCharTok{$}\NormalTok{westcoast }\OtherTok{\textless{}{-}} \FunctionTok{sapply}\NormalTok{(breaches}\SpecialCharTok{$}\NormalTok{State, westcoast\_check)}
\FunctionTok{head}\NormalTok{(breaches)}
\end{Highlighting}
\end{Shaded}

\begin{verbatim}
## # A tibble: 6 x 16
##      X1 Number Name_of_Covered_Entity  State Business_Associat~ Individuals_Aff~
##   <dbl>  <dbl> <chr>                   <fct> <chr>                         <int>
## 1     1      0 Brooke Army Medical Ce~ TX    <NA>                           1000
## 2     2      1 Mid America Kidney Sto~ MO    <NA>                           1000
## 3     3      2 Alaska Department of H~ AK    <NA>                            501
## 4     4      3 Health Services for Ch~ DC    <NA>                           3800
## 5     5      4 L. Douglas Carlson, M.~ CA    <NA>                           5257
## 6     6      5 David I. Cohen, MD      CA    <NA>                            857
## # ... with 10 more variables: Date_of_Breach <chr>, Type_of_Breach <chr>,
## #   Location_of_Breached_Information <chr>, Date_Posted_or_Updated <date>,
## #   Summary <chr>, breach_start <date>, breach_end <date>, year <dbl>,
## #   northeast <lgl>, westcoast <lgl>
\end{verbatim}

\begin{Shaded}
\begin{Highlighting}[]
\NormalTok{midwest\_list }\OtherTok{\textless{}{-}} \FunctionTok{c}\NormalTok{(}\StringTok{"ND"}\NormalTok{, }\StringTok{"SD"}\NormalTok{, }\StringTok{"NE"}\NormalTok{, }\StringTok{"KS"}\NormalTok{, }\StringTok{"MO"}\NormalTok{, }\StringTok{"IA"}\NormalTok{, }\StringTok{"MN"}\NormalTok{, }\StringTok{"WI"}\NormalTok{, }\StringTok{"MI"}\NormalTok{, }\StringTok{"IL"}\NormalTok{, }\StringTok{"IN"}\NormalTok{, }\StringTok{"OH"}\NormalTok{, }\StringTok{"MT"}\NormalTok{)}
\CommentTok{\#function to determine if the states are in the midwest}
\NormalTok{midwest\_check }\OtherTok{\textless{}{-}} \ControlFlowTok{function}\NormalTok{(x) \{}
  \ControlFlowTok{if}\NormalTok{(}\FunctionTok{is.na}\NormalTok{(x))\{}
    \FunctionTok{return}\NormalTok{(}\ConstantTok{NA}\NormalTok{)}
\NormalTok{  \}}
  \ControlFlowTok{else} \ControlFlowTok{if}\NormalTok{(x }\SpecialCharTok{\%in\%}\NormalTok{ midwest\_list)\{}
    \FunctionTok{return}\NormalTok{(}\ConstantTok{TRUE}\NormalTok{)}
\NormalTok{  \}}
  \ControlFlowTok{else}\NormalTok{\{}
    \FunctionTok{return}\NormalTok{(}\ConstantTok{FALSE}\NormalTok{)}
\NormalTok{  \}}
\NormalTok{\}}

\NormalTok{breaches}\SpecialCharTok{$}\NormalTok{midwest }\OtherTok{\textless{}{-}} \FunctionTok{sapply}\NormalTok{(breaches}\SpecialCharTok{$}\NormalTok{State, midwest\_check)}
\FunctionTok{head}\NormalTok{(breaches)}
\end{Highlighting}
\end{Shaded}

\begin{verbatim}
## # A tibble: 6 x 17
##      X1 Number Name_of_Covered_Entity  State Business_Associat~ Individuals_Aff~
##   <dbl>  <dbl> <chr>                   <fct> <chr>                         <int>
## 1     1      0 Brooke Army Medical Ce~ TX    <NA>                           1000
## 2     2      1 Mid America Kidney Sto~ MO    <NA>                           1000
## 3     3      2 Alaska Department of H~ AK    <NA>                            501
## 4     4      3 Health Services for Ch~ DC    <NA>                           3800
## 5     5      4 L. Douglas Carlson, M.~ CA    <NA>                           5257
## 6     6      5 David I. Cohen, MD      CA    <NA>                            857
## # ... with 11 more variables: Date_of_Breach <chr>, Type_of_Breach <chr>,
## #   Location_of_Breached_Information <chr>, Date_Posted_or_Updated <date>,
## #   Summary <chr>, breach_start <date>, breach_end <date>, year <dbl>,
## #   northeast <lgl>, westcoast <lgl>, midwest <lgl>
\end{verbatim}

\begin{Shaded}
\begin{Highlighting}[]
\NormalTok{southwest\_list }\OtherTok{\textless{}{-}} \FunctionTok{c}\NormalTok{(}\StringTok{"AZ"}\NormalTok{, }\StringTok{"NM"}\NormalTok{, }\StringTok{"OK"}\NormalTok{, }\StringTok{"TX"}\NormalTok{)}
\NormalTok{other\_list }\OtherTok{\textless{}{-}} \FunctionTok{c}\NormalTok{(}\StringTok{"DC"}\NormalTok{, }\StringTok{"PR"}\NormalTok{)}
\end{Highlighting}
\end{Shaded}

\begin{Shaded}
\begin{Highlighting}[]
\NormalTok{south\_list }\OtherTok{\textless{}{-}} \FunctionTok{c}\NormalTok{(}\StringTok{"MD"}\NormalTok{, }\StringTok{"DE"}\NormalTok{, }\StringTok{"VA"}\NormalTok{, }\StringTok{"WV"}\NormalTok{, }\StringTok{"KY"}\NormalTok{, }\StringTok{"TN"}\NormalTok{, }\StringTok{"NC"}\NormalTok{, }\StringTok{"SC"}\NormalTok{, }\StringTok{"FL"}\NormalTok{, }\StringTok{"GA"}\NormalTok{, }\StringTok{"AL"}\NormalTok{, }\StringTok{"MS"}\NormalTok{, }\StringTok{"LA"}\NormalTok{, }\StringTok{"AK"}\NormalTok{, }\StringTok{"OK"}\NormalTok{, }\StringTok{"TX"}\NormalTok{, }\StringTok{"DC"}\NormalTok{, }\StringTok{"AR"}\NormalTok{)}
\CommentTok{\#function to determine if the states are in the south}
\NormalTok{south\_check }\OtherTok{\textless{}{-}} \ControlFlowTok{function}\NormalTok{(x) \{}
  \ControlFlowTok{if}\NormalTok{(}\FunctionTok{is.na}\NormalTok{(x))\{}
    \FunctionTok{return}\NormalTok{(}\ConstantTok{NA}\NormalTok{)}
\NormalTok{  \}}
  \ControlFlowTok{else} \ControlFlowTok{if}\NormalTok{(x }\SpecialCharTok{\%in\%}\NormalTok{ south\_list)\{}
    \FunctionTok{return}\NormalTok{(}\ConstantTok{TRUE}\NormalTok{)}
\NormalTok{  \}}
  \ControlFlowTok{else}\NormalTok{\{}
    \FunctionTok{return}\NormalTok{(}\ConstantTok{FALSE}\NormalTok{)}
\NormalTok{  \}}
\NormalTok{\}}

\NormalTok{breaches}\SpecialCharTok{$}\NormalTok{south }\OtherTok{\textless{}{-}} \FunctionTok{sapply}\NormalTok{(breaches}\SpecialCharTok{$}\NormalTok{State, south\_check)}
\FunctionTok{head}\NormalTok{(breaches)}
\end{Highlighting}
\end{Shaded}

\begin{verbatim}
## # A tibble: 6 x 18
##      X1 Number Name_of_Covered_Entity  State Business_Associat~ Individuals_Aff~
##   <dbl>  <dbl> <chr>                   <fct> <chr>                         <int>
## 1     1      0 Brooke Army Medical Ce~ TX    <NA>                           1000
## 2     2      1 Mid America Kidney Sto~ MO    <NA>                           1000
## 3     3      2 Alaska Department of H~ AK    <NA>                            501
## 4     4      3 Health Services for Ch~ DC    <NA>                           3800
## 5     5      4 L. Douglas Carlson, M.~ CA    <NA>                           5257
## 6     6      5 David I. Cohen, MD      CA    <NA>                            857
## # ... with 12 more variables: Date_of_Breach <chr>, Type_of_Breach <chr>,
## #   Location_of_Breached_Information <chr>, Date_Posted_or_Updated <date>,
## #   Summary <chr>, breach_start <date>, breach_end <date>, year <dbl>,
## #   northeast <lgl>, westcoast <lgl>, midwest <lgl>, south <lgl>
\end{verbatim}

\begin{Shaded}
\begin{Highlighting}[]
\NormalTok{region\_check }\OtherTok{\textless{}{-}} \ControlFlowTok{function}\NormalTok{(x) \{}
  \ControlFlowTok{if}\NormalTok{(}\FunctionTok{is.na}\NormalTok{(x))\{}
    \FunctionTok{return}\NormalTok{(}\ConstantTok{NA}\NormalTok{)}
\NormalTok{  \}}
  \ControlFlowTok{else} \ControlFlowTok{if}\NormalTok{(x }\SpecialCharTok{\%in\%}\NormalTok{ westcoast\_list)\{}
    \FunctionTok{return}\NormalTok{(}\StringTok{"westcoast"}\NormalTok{)}
\NormalTok{  \}}
  \ControlFlowTok{else} \ControlFlowTok{if}\NormalTok{(x }\SpecialCharTok{\%in\%}\NormalTok{ northeast\_list)\{}
    \FunctionTok{return}\NormalTok{(}\StringTok{"northeast"}\NormalTok{)}
\NormalTok{  \}}
  \ControlFlowTok{else} \ControlFlowTok{if}\NormalTok{(x }\SpecialCharTok{\%in\%}\NormalTok{ midwest\_list)\{}
    \FunctionTok{return}\NormalTok{(}\StringTok{"midwest"}\NormalTok{)}
\NormalTok{  \}}
  \ControlFlowTok{else} \ControlFlowTok{if}\NormalTok{(x }\SpecialCharTok{\%in\%}\NormalTok{ south\_list)\{}
    \FunctionTok{return}\NormalTok{(}\StringTok{"south"}\NormalTok{)}
\NormalTok{  \}}
  \ControlFlowTok{else}\NormalTok{\{}
    \FunctionTok{return}\NormalTok{(}\StringTok{"other"}\NormalTok{)}
\NormalTok{  \}}
\NormalTok{\}}


\NormalTok{breaches}\SpecialCharTok{$}\NormalTok{region }\OtherTok{\textless{}{-}} \FunctionTok{sapply}\NormalTok{(breaches}\SpecialCharTok{$}\NormalTok{State, region\_check)}

\NormalTok{region\_levels }\OtherTok{=} \FunctionTok{c}\NormalTok{(}\StringTok{"northeast"}\NormalTok{, }\StringTok{"midwest"}\NormalTok{, }\StringTok{"south"}\NormalTok{, }\StringTok{"westcoast"}\NormalTok{, }\StringTok{"other"}\NormalTok{)}

\NormalTok{breaches}\SpecialCharTok{$}\NormalTok{region }\OtherTok{\textless{}{-}} \FunctionTok{factor}\NormalTok{(breaches}\SpecialCharTok{$}\NormalTok{region, }\AttributeTok{levels=}\NormalTok{ region\_levels)}
\end{Highlighting}
\end{Shaded}

\begin{Shaded}
\begin{Highlighting}[]
\NormalTok{breaches }\SpecialCharTok{\%\textgreater{}\%}
  \FunctionTok{ggplot}\NormalTok{(}\FunctionTok{aes}\NormalTok{(}\AttributeTok{x =}\NormalTok{ region, }\AttributeTok{y =}\NormalTok{ Individuals\_Affected, }\AttributeTok{fill =}\NormalTok{ region)) }\SpecialCharTok{+}
  \FunctionTok{geom\_col}\NormalTok{()}\SpecialCharTok{+}
  \FunctionTok{ylab}\NormalTok{(}\StringTok{"Number of Individuals Affected by Breaches"}\NormalTok{)}
\end{Highlighting}
\end{Shaded}

\includegraphics{CourtneyKennedy_files/figure-latex/unnamed-chunk-16-1.pdf}

\textbf{- Describe and discuss why they are needed and how you plan to
use them.} northeast, westcoast, midwest, and south, are all a logical
variable. They are needed in exploring the distribution of breaches per
state in different regions of the US. Logical variables are used since
logical are simpler to evaluate when larger factor data is converted
into two groups. I plan to use the variables to compare individuals
affected by their location.

The region variable sorts the US into regions based on the state the
breach occured in. I am planning on using the region variable to compare
the categorical states to the individuals affected in boxplots and bar
graphs.

\begin{Shaded}
\begin{Highlighting}[]
\NormalTok{northeast\_bar }\OtherTok{\textless{}{-}} 
\NormalTok{breaches }\SpecialCharTok{\%\textgreater{}\%}
  \FunctionTok{ggplot}\NormalTok{(}\FunctionTok{aes}\NormalTok{(}\AttributeTok{x=}\NormalTok{northeast, }\AttributeTok{fill =}\NormalTok{ northeast)) }\SpecialCharTok{+}
  \FunctionTok{geom\_bar}\NormalTok{()}

\NormalTok{midwest\_bar }\OtherTok{\textless{}{-}} 
\NormalTok{breaches }\SpecialCharTok{\%\textgreater{}\%}
  \FunctionTok{ggplot}\NormalTok{(}\FunctionTok{aes}\NormalTok{(}\AttributeTok{x=}\NormalTok{midwest, }\AttributeTok{fill =}\NormalTok{ midwest)) }\SpecialCharTok{+}
  \FunctionTok{geom\_bar}\NormalTok{()}

\NormalTok{westcoast\_bar }\OtherTok{\textless{}{-}} 
\NormalTok{breaches }\SpecialCharTok{\%\textgreater{}\%}
  \FunctionTok{ggplot}\NormalTok{(}\FunctionTok{aes}\NormalTok{(}\AttributeTok{x=}\NormalTok{westcoast, }\AttributeTok{fill =}\NormalTok{ westcoast)) }\SpecialCharTok{+}
  \FunctionTok{geom\_bar}\NormalTok{()}

\NormalTok{south\_bar }\OtherTok{\textless{}{-}} 
\NormalTok{breaches }\SpecialCharTok{\%\textgreater{}\%}
  \FunctionTok{ggplot}\NormalTok{(}\FunctionTok{aes}\NormalTok{(}\AttributeTok{x=}\NormalTok{south, }\AttributeTok{fill =}\NormalTok{ south)) }\SpecialCharTok{+}
  \FunctionTok{geom\_bar}\NormalTok{()}

\FunctionTok{ggarrange}\NormalTok{(northeast\_bar, midwest\_bar, westcoast\_bar, south\_bar, }\AttributeTok{nrow =} \DecValTok{2}\NormalTok{, }\AttributeTok{ncol =} \DecValTok{2}\NormalTok{)}
\end{Highlighting}
\end{Shaded}

\includegraphics{CourtneyKennedy_files/figure-latex/unnamed-chunk-17-1.pdf}
\# Individuals\_affected Variable

\hypertarget{visualising-distributions-barcharts-histograms}{%
\subsection{Visualising distributions (Barcharts,
Histograms)}\label{visualising-distributions-barcharts-histograms}}

\begin{Shaded}
\begin{Highlighting}[]
\NormalTok{indiv\_box }\OtherTok{\textless{}{-}}\NormalTok{ breaches }\SpecialCharTok{\%\textgreater{}\%}
  \FunctionTok{ggplot}\NormalTok{(}\FunctionTok{aes}\NormalTok{(}\AttributeTok{x=}\NormalTok{Individuals\_Affected)) }\SpecialCharTok{+}
  \FunctionTok{geom\_boxplot}\NormalTok{()}

\NormalTok{indiv\_hist }\OtherTok{\textless{}{-}}\NormalTok{ breaches }\SpecialCharTok{\%\textgreater{}\%}
  \FunctionTok{ggplot}\NormalTok{(}\FunctionTok{aes}\NormalTok{(}\AttributeTok{x=}\NormalTok{Individuals\_Affected)) }\SpecialCharTok{+}
  \FunctionTok{geom\_histogram}\NormalTok{()}

\NormalTok{indiv\_box\_zoom }\OtherTok{\textless{}{-}}\NormalTok{ breaches }\SpecialCharTok{\%\textgreater{}\%}
  \FunctionTok{ggplot}\NormalTok{(}\FunctionTok{aes}\NormalTok{(}\AttributeTok{x=}\NormalTok{Individuals\_Affected)) }\SpecialCharTok{+}
  \FunctionTok{geom\_boxplot}\NormalTok{()}\SpecialCharTok{+}
  \FunctionTok{xlim}\NormalTok{(}\DecValTok{0}\NormalTok{, }\DecValTok{35000}\NormalTok{) }\SpecialCharTok{+} 
  \FunctionTok{labs}\NormalTok{(}\AttributeTok{title =} \StringTok{"0 to 35,000 zoom in"}\NormalTok{)}

\NormalTok{indiv\_hist\_zoom }\OtherTok{\textless{}{-}}\NormalTok{ breaches }\SpecialCharTok{\%\textgreater{}\%}
  \FunctionTok{ggplot}\NormalTok{(}\FunctionTok{aes}\NormalTok{(}\AttributeTok{x=}\NormalTok{Individuals\_Affected)) }\SpecialCharTok{+}
  \FunctionTok{geom\_histogram}\NormalTok{() }\SpecialCharTok{+} 
  \FunctionTok{xlim}\NormalTok{(}\DecValTok{0}\NormalTok{, }\DecValTok{35000}\NormalTok{) }\SpecialCharTok{+} 
  \FunctionTok{labs}\NormalTok{(}\AttributeTok{title =} \StringTok{"0 to 35,000 zoom in"}\NormalTok{)}

\FunctionTok{ggarrange}\NormalTok{(indiv\_box, indiv\_hist, indiv\_box\_zoom, indiv\_hist\_zoom,  }\AttributeTok{nrow =} \DecValTok{2}\NormalTok{, }\AttributeTok{ncol =} \DecValTok{2}\NormalTok{)}
\end{Highlighting}
\end{Shaded}

\begin{verbatim}
## `stat_bin()` using `bins = 30`. Pick better value with `binwidth`.
\end{verbatim}

\begin{verbatim}
## Warning: Removed 69 rows containing non-finite values (stat_boxplot).
\end{verbatim}

\begin{verbatim}
## `stat_bin()` using `bins = 30`. Pick better value with `binwidth`.
\end{verbatim}

\begin{verbatim}
## Warning: Removed 69 rows containing non-finite values (stat_bin).
\end{verbatim}

\begin{verbatim}
## Warning: Removed 2 rows containing missing values (geom_bar).
\end{verbatim}

\includegraphics{CourtneyKennedy_files/figure-latex/unnamed-chunk-18-1.pdf}

\begin{Shaded}
\begin{Highlighting}[]
\FunctionTok{summary}\NormalTok{(breaches)}
\end{Highlighting}
\end{Shaded}

\begin{verbatim}
##        X1             Number       Name_of_Covered_Entity     State    
##  Min.   :   1.0   Min.   :   0.0   Length:1055            CA     :113  
##  1st Qu.: 264.5   1st Qu.: 263.5   Class :character       TX     : 83  
##  Median : 528.0   Median : 527.0   Mode  :character       FL     : 66  
##  Mean   : 528.0   Mean   : 527.0                          NY     : 58  
##  3rd Qu.: 791.5   3rd Qu.: 790.5                          IL     : 49  
##  Max.   :1055.0   Max.   :1054.0                          PA     : 40  
##                                                           (Other):646  
##  Business_Associate_Involved Individuals_Affected Date_of_Breach    
##  Length:1055                 Min.   :    500      Length:1055       
##  Class :character            1st Qu.:   1000      Class :character  
##  Mode  :character            Median :   2300      Mode  :character  
##                              Mean   :  30262                        
##                              3rd Qu.:   6941                        
##                              Max.   :4900000                        
##                                                                     
##  Type_of_Breach     Location_of_Breached_Information Date_Posted_or_Updated
##  Length:1055        Length:1055                      Min.   :2014-01-23    
##  Class :character   Class :character                 1st Qu.:2014-01-23    
##  Mode  :character   Mode  :character                 Median :2014-01-23    
##                                                      Mean   :2014-02-23    
##                                                      3rd Qu.:2014-03-24    
##                                                      Max.   :2014-06-30    
##                                                                            
##    Summary           breach_start          breach_end              year     
##  Length:1055        Min.   :1997-01-01   Min.   :2007-06-14   Min.   :1997  
##  Class :character   1st Qu.:2010-11-08   1st Qu.:2012-04-22   1st Qu.:2010  
##  Mode  :character   Median :2012-01-11   Median :2012-10-29   Median :2012  
##                     Mean   :2011-12-09   Mean   :2012-10-28   Mean   :2011  
##                     3rd Qu.:2013-03-07   3rd Qu.:2013-05-29   3rd Qu.:2013  
##                     Max.   :2014-06-02   Max.   :2013-11-30   Max.   :2014  
##                                          NA's   :910                        
##  northeast       westcoast        midwest          south        
##  Mode :logical   Mode :logical   Mode :logical   Mode :logical  
##  FALSE:854       FALSE:828       FALSE:815       FALSE:674      
##  TRUE :201       TRUE :227       TRUE :240       TRUE :381      
##                                                                 
##                                                                 
##                                                                 
##                                                                 
##        region   
##  northeast:201  
##  midwest  :240  
##  south    :355  
##  westcoast:227  
##  other    : 32  
##                 
## 
\end{verbatim}

\begin{Shaded}
\begin{Highlighting}[]
\FunctionTok{IQR}\NormalTok{(breaches}\SpecialCharTok{$}\NormalTok{Individuals\_Affected, }\AttributeTok{na.rm =} \ConstantTok{TRUE}\NormalTok{)}
\end{Highlighting}
\end{Shaded}

\begin{verbatim}
## [1] 5941
\end{verbatim}

\textbf{- Which values are the most common? Why?}

The values in the IQR are the most common which ranges from 500 to 6941
people. This can be seen in the histogram since the peak is centered
around 2300 people, which is the median. The majority of the values fall
in this range and therefore they are statistically the most common. This
can be interpreted that in most data breaches the number of individuals
affected is usually between 500 to around 7000 people.

\textbf{- Which values are rare? Why? Does that match your
expectations?}

Values above 1 million are more rare. This does match my expectations
since large breaches are less common, and therefore breaches with
indiviuals affected being above 1 million are more rare.

\textbf{- Can you see any unusual patterns? What might explain them?}

There is no cycle pattern present in the individuals affected data. The
only slightly unusual pattern is that there is a strong right skew.
There are some very large values for individuals affected that drag the
mean up, and therefore the data is very right skewed. Overall the median
is a better reference to the middle of the data than the mean. This
right skew is caused by a few data breaches that had very high numbers
of indivuals affected.

\textbf{- Are there clusters in the data? If so,} There is a large
grouping of data at about 5000 individuals affected and below. There is
not another large grouping however that could be defined as a cluster.

\textbf{- How are the observations within each cluster similar to or
different from each other?}

The observations in the low individuals affected cluster all come from
different states, and there isn't an obvious connection between the
points.

\textbf{- How can you explain or describe the clusters?}

The cluster can possibly be explained by the fact that most breaches are
on the smaller side, and that breaches that affect a lot of people are
harder to pull off and therefore more rare.

\hypertarget{unusual-values-1}{%
\paragraph{3.1.2.2 Unusual values}\label{unusual-values-1}}

\textbf{- Describe and demonstrate how you determine if there are
unusual values in the data. E.g. too large, too small, negative, etc.}

There are no negative values for individuals affected, and there are two
very large values, above 3 million. I filtered for both situations to
confirm this result of unusual values.

\begin{Shaded}
\begin{Highlighting}[]
\NormalTok{neg\_indiv }\OtherTok{\textless{}{-}}\NormalTok{ breaches }\SpecialCharTok{\%\textgreater{}\%}
  \FunctionTok{filter}\NormalTok{(Individuals\_Affected }\SpecialCharTok{\textless{}} \DecValTok{0}\NormalTok{)}

\NormalTok{neg\_indiv}
\end{Highlighting}
\end{Shaded}

\begin{verbatim}
## # A tibble: 0 x 19
## # ... with 19 variables: X1 <dbl>, Number <dbl>, Name_of_Covered_Entity <chr>,
## #   State <fct>, Business_Associate_Involved <chr>, Individuals_Affected <int>,
## #   Date_of_Breach <chr>, Type_of_Breach <chr>,
## #   Location_of_Breached_Information <chr>, Date_Posted_or_Updated <date>,
## #   Summary <chr>, breach_start <date>, breach_end <date>, year <dbl>,
## #   northeast <lgl>, westcoast <lgl>, midwest <lgl>, south <lgl>, region <fct>
\end{verbatim}

\begin{Shaded}
\begin{Highlighting}[]
\NormalTok{large\_indiv }\OtherTok{\textless{}{-}}\NormalTok{ breaches }\SpecialCharTok{\%\textgreater{}\%}
  \FunctionTok{filter}\NormalTok{(Individuals\_Affected }\SpecialCharTok{\textgreater{}} \DecValTok{3000000}\NormalTok{)}

\NormalTok{large\_indiv}
\end{Highlighting}
\end{Shaded}

\begin{verbatim}
## # A tibble: 2 x 19
##      X1 Number Name_of_Covered_Entity State Business_Associate~ Individuals_Aff~
##   <dbl>  <dbl> <chr>                  <fct> <chr>                          <int>
## 1   410    409 TRICARE Management Ac~ VA    Science Applicatio~          4900000
## 2   800    799 Advocate Health and H~ IL    <NA>                         4029530
## # ... with 13 more variables: Date_of_Breach <chr>, Type_of_Breach <chr>,
## #   Location_of_Breached_Information <chr>, Date_Posted_or_Updated <date>,
## #   Summary <chr>, breach_start <date>, breach_end <date>, year <dbl>,
## #   northeast <lgl>, westcoast <lgl>, midwest <lgl>, south <lgl>, region <fct>
\end{verbatim}

\textbf{- Describe and demonstrate how you determine if they are
outliers.}

An outlier is 1.5 times the interquartile range away from either the
lower or upper quartile. In order to determine if any of the indivuals
affected values are outliers the interquartile range, first quartile,
and third quartile need to be calculated. The indivduals affected data
then has to be filtered for values that are less than the first quartile
minus the IQR times 1.5 and values that are greater than the third
quartile plus the IQR times 1.5. The 129 outliers can be seen in the
outlier list.

\begin{Shaded}
\begin{Highlighting}[]
\NormalTok{stdev }\OtherTok{\textless{}{-}}  \FunctionTok{sd}\NormalTok{(breaches}\SpecialCharTok{$}\NormalTok{Individuals\_Affected, }\AttributeTok{na.rm =} \ConstantTok{TRUE}\NormalTok{)}
\NormalTok{stdev}
\end{Highlighting}
\end{Shaded}

\begin{verbatim}
## [1] 227859.8
\end{verbatim}

\begin{Shaded}
\begin{Highlighting}[]
\NormalTok{innerQ }\OtherTok{\textless{}{-}}  \FunctionTok{IQR}\NormalTok{(breaches}\SpecialCharTok{$}\NormalTok{Individuals\_Affected, }\AttributeTok{na.rm =} \ConstantTok{TRUE}\NormalTok{)}
\NormalTok{innerQ}
\end{Highlighting}
\end{Shaded}

\begin{verbatim}
## [1] 5941
\end{verbatim}

\begin{Shaded}
\begin{Highlighting}[]
\NormalTok{firstQ }\OtherTok{\textless{}{-}} \FunctionTok{quantile}\NormalTok{(breaches}\SpecialCharTok{$}\NormalTok{Individuals\_Affected, }\FloatTok{0.25}\NormalTok{, }\AttributeTok{na.rm =} \ConstantTok{TRUE}\NormalTok{)}
\NormalTok{firstQ }\OtherTok{\textless{}{-}}\NormalTok{ firstQ[[}\DecValTok{1}\NormalTok{]]}

\NormalTok{thirdQ }\OtherTok{\textless{}{-}} \FunctionTok{quantile}\NormalTok{(breaches}\SpecialCharTok{$}\NormalTok{Individuals\_Affected, }\FloatTok{0.75}\NormalTok{, }\AttributeTok{na.rm =} \ConstantTok{TRUE}\NormalTok{)}
\NormalTok{thirdQ }\OtherTok{\textless{}{-}}\NormalTok{ thirdQ[[}\DecValTok{1}\NormalTok{]]}

\NormalTok{outlier\_list }\OtherTok{\textless{}{-}}\NormalTok{ breaches }\SpecialCharTok{\%\textgreater{}\%}
  \FunctionTok{filter}\NormalTok{(Individuals\_Affected }\SpecialCharTok{\textless{}}\NormalTok{ (firstQ }\SpecialCharTok{{-}}\NormalTok{ innerQ }\SpecialCharTok{*} \FloatTok{1.5}\NormalTok{) }\SpecialCharTok{|} 
\NormalTok{        Individuals\_Affected }\SpecialCharTok{\textgreater{}}\NormalTok{ (thirdQ }\SpecialCharTok{+}\NormalTok{ innerQ }\SpecialCharTok{*} \FloatTok{1.5}\NormalTok{))}

\NormalTok{outlier\_list}
\end{Highlighting}
\end{Shaded}

\begin{verbatim}
## # A tibble: 129 x 19
##       X1 Number Name_of_Covered_Enti~ State Business_Associate~ Individuals_Aff~
##    <dbl>  <dbl> <chr>                 <fct> <chr>                          <int>
##  1    13     12 "Universal American"  NY    Democracy Data & C~            83000
##  2    50     49 "Ernest T. Bice, Jr.~ TX    <NA>                           21000
##  3    59     58 "Providence Hospital" MI    <NA>                           83945
##  4    64     63 "Affinity Health Pla~ NY    <NA>                          344579
##  5    66     65 "Praxair Healthcare ~ CT    <NA>                           54165
##  6    70     69 "St. Joseph Heritage~ CA    <NA>                           22012
##  7    76     75 "Emergency Healthcar~ IL    Millennium Medical~           180111
##  8    81     80 "Silicon Valley Eyec~ CA    <NA>                           40000
##  9    91     90 "Cincinnati Children~ OH    <NA>                           60998
## 10    93     92 "AvMed, Inc."         FL    <NA>                         1220000
## # ... with 119 more rows, and 13 more variables: Date_of_Breach <chr>,
## #   Type_of_Breach <chr>, Location_of_Breached_Information <chr>,
## #   Date_Posted_or_Updated <date>, Summary <chr>, breach_start <date>,
## #   breach_end <date>, year <dbl>, northeast <lgl>, westcoast <lgl>,
## #   midwest <lgl>, south <lgl>, region <fct>
\end{verbatim}

\textbf{- Show how do your distributions look like with and without the
unusual values.}

\begin{Shaded}
\begin{Highlighting}[]
\NormalTok{outliers\_removed }\OtherTok{\textless{}{-}}\NormalTok{ breaches }\SpecialCharTok{\%\textgreater{}\%}
  \FunctionTok{filter}\NormalTok{(}\SpecialCharTok{!}\NormalTok{Individuals\_Affected }\SpecialCharTok{\%in\%}\NormalTok{ outlier\_list}\SpecialCharTok{$}\NormalTok{Individuals\_Affected) }\SpecialCharTok{\%\textgreater{}\%}
  \FunctionTok{ggplot}\NormalTok{(}\FunctionTok{aes}\NormalTok{(}\AttributeTok{x=}\NormalTok{Individuals\_Affected))}\SpecialCharTok{+}
  \FunctionTok{geom\_histogram}\NormalTok{() }\SpecialCharTok{+}
  \FunctionTok{labs}\NormalTok{(}\AttributeTok{title =} \StringTok{"Outliers Removed"}\NormalTok{)}

\NormalTok{outliers\_included }\OtherTok{\textless{}{-}}\NormalTok{ breaches }\SpecialCharTok{\%\textgreater{}\%}
  \FunctionTok{ggplot}\NormalTok{(}\FunctionTok{aes}\NormalTok{(}\AttributeTok{x=}\NormalTok{Individuals\_Affected)) }\SpecialCharTok{+}
  \FunctionTok{geom\_histogram}\NormalTok{()}\SpecialCharTok{+}
  \FunctionTok{labs}\NormalTok{(}\AttributeTok{title =} \StringTok{"Outliers Included"}\NormalTok{)}

\FunctionTok{ggarrange}\NormalTok{(outliers\_removed, outliers\_included, }\AttributeTok{nrow =} \DecValTok{2}\NormalTok{)}
\end{Highlighting}
\end{Shaded}

\begin{verbatim}
## `stat_bin()` using `bins = 30`. Pick better value with `binwidth`.
## `stat_bin()` using `bins = 30`. Pick better value with `binwidth`.
\end{verbatim}

\includegraphics{CourtneyKennedy_files/figure-latex/unnamed-chunk-21-1.pdf}

\textbf{- Discuss whether or not you need to remove unusual values and
why.}

The unusual values should not be removed because since the high
individuals affected values will most likely give the most insight into
cyber security issues.

\hypertarget{missing-values-1}{%
\paragraph{3.1.2.3 Missing values}\label{missing-values-1}}

\textbf{- Does this variable include missing values? Demonstrate how you
determine that.}

No there are no missing values. The method is.na with the column name
can be used and then the vector returned can be turned into a data frame
that represents the number of NA values (TRUE) and non NA values
(FALSE). It can also be confirmed by calling summary() on the
Individuals Affected variable, which also shows that there are no NA
values in the Individuals affected variable.

\begin{Shaded}
\begin{Highlighting}[]
\NormalTok{missing }\OtherTok{\textless{}{-}} \FunctionTok{is.na}\NormalTok{(breaches}\SpecialCharTok{$}\NormalTok{Individuals\_Affected)}

\NormalTok{num\_missing }\OtherTok{\textless{}{-}} \FunctionTok{as.data.frame}\NormalTok{(}\FunctionTok{table}\NormalTok{(missing))}

\NormalTok{num\_missing}
\end{Highlighting}
\end{Shaded}

\begin{verbatim}
##   missing Freq
## 1   FALSE 1055
\end{verbatim}

\begin{Shaded}
\begin{Highlighting}[]
\FunctionTok{summary}\NormalTok{(breaches}\SpecialCharTok{$}\NormalTok{Individuals\_Affected)}
\end{Highlighting}
\end{Shaded}

\begin{verbatim}
##    Min. 1st Qu.  Median    Mean 3rd Qu.    Max. 
##     500    1000    2300   30262    6941 4900000
\end{verbatim}

\textbf{- Demonstrate and discuss how you handle the missing values.
E.g., removing, replacing with a constant value, or a value based on the
distribution, etc.}

There are no missing values

\textbf{- Show how your data looks in each case after handling missing
values.Describe and discuss the distribution.}

There are no missing values. Refer to histogram and boxplots above for
distribution.

\hypertarget{does-converting-the-type-of-this-variable-help-exploring-the-distribution-of-its-values-or-identifying-outliers-or-missing-values-3-1}{%
\paragraph{3.1.2.4 Does converting the type of this variable help
exploring the distribution of its values or identifying outliers or
missing values?
(3)}\label{does-converting-the-type-of-this-variable-help-exploring-the-distribution-of-its-values-or-identifying-outliers-or-missing-values-3-1}}

Yes converting Individuals affected to a logical may be helpful in
exploring the distribution of its values or identifying outliers or
missing values since logical are simpler to evaluate when larger
continuous data is converted into two groups.

\textbf{- What type can the variable be converted to?}

Individuals affected is of type integer, but it can converted to a
logical. By making the value of Individuals affected TRUE when the value
is greater than 20,000 and FALSE when the value is lower than than
20,000, we can see if the Inidviduals Affected level is considered high
or not. Converting Individuals Affected to a logical is a simpler way to
interpret Individuals values. The converted Individuals Affected type is
saved as a new variable Large\_Affected.

\begin{Shaded}
\begin{Highlighting}[]
\NormalTok{high\_check }\OtherTok{\textless{}{-}} \ControlFlowTok{function}\NormalTok{(x) \{}
  \ControlFlowTok{if}\NormalTok{(}\FunctionTok{is.na}\NormalTok{(x))\{}
    \FunctionTok{return}\NormalTok{(}\ConstantTok{NA}\NormalTok{)}
\NormalTok{  \}}
  \ControlFlowTok{else} \ControlFlowTok{if}\NormalTok{(x }\SpecialCharTok{\textgreater{}=} \DecValTok{20000}\NormalTok{)\{}
    \FunctionTok{return}\NormalTok{(}\ConstantTok{TRUE}\NormalTok{)}
\NormalTok{  \}}
  \ControlFlowTok{else}\NormalTok{\{}
    \FunctionTok{return}\NormalTok{(}\ConstantTok{FALSE}\NormalTok{)}
\NormalTok{  \}}
\NormalTok{\}}

\NormalTok{breaches}\SpecialCharTok{$}\NormalTok{large\_affected }\OtherTok{\textless{}{-}} \FunctionTok{sapply}\NormalTok{(breaches}\SpecialCharTok{$}\NormalTok{Individuals\_Affected, high\_check)}
\FunctionTok{head}\NormalTok{(breaches)}
\end{Highlighting}
\end{Shaded}

\begin{verbatim}
## # A tibble: 6 x 20
##      X1 Number Name_of_Covered_Entity  State Business_Associat~ Individuals_Aff~
##   <dbl>  <dbl> <chr>                   <fct> <chr>                         <int>
## 1     1      0 Brooke Army Medical Ce~ TX    <NA>                           1000
## 2     2      1 Mid America Kidney Sto~ MO    <NA>                           1000
## 3     3      2 Alaska Department of H~ AK    <NA>                            501
## 4     4      3 Health Services for Ch~ DC    <NA>                           3800
## 5     5      4 L. Douglas Carlson, M.~ CA    <NA>                           5257
## 6     6      5 David I. Cohen, MD      CA    <NA>                            857
## # ... with 14 more variables: Date_of_Breach <chr>, Type_of_Breach <chr>,
## #   Location_of_Breached_Information <chr>, Date_Posted_or_Updated <date>,
## #   Summary <chr>, breach_start <date>, breach_end <date>, year <dbl>,
## #   northeast <lgl>, westcoast <lgl>, midwest <lgl>, south <lgl>, region <fct>,
## #   large_affected <lgl>
\end{verbatim}

\textbf{- How will the distribution look? Please demonstrate with
appropriate plots.}

From plotting the converted logical Individuals Affected variable as a
bar graph, we can see that the majority of the breaches were above
20,000 people affected. We can also see that there are no NA values,
which confirms the analysis done earlier.

\begin{Shaded}
\begin{Highlighting}[]
\NormalTok{breaches }\SpecialCharTok{\%\textgreater{}\%}
  \FunctionTok{ggplot}\NormalTok{(}\FunctionTok{aes}\NormalTok{(}\AttributeTok{x=}\NormalTok{large\_affected, }\AttributeTok{fill =}\NormalTok{ large\_affected)) }\SpecialCharTok{+}
  \FunctionTok{geom\_bar}\NormalTok{() }\SpecialCharTok{+}
  \FunctionTok{labs}\NormalTok{(}\AttributeTok{x =} \StringTok{"Breach over 20,000 Individuals"}\NormalTok{, }\AttributeTok{title =} \StringTok{"Distribution of large and small breaches"}\NormalTok{)}
\end{Highlighting}
\end{Shaded}

\includegraphics{CourtneyKennedy_files/figure-latex/unnamed-chunk-24-1.pdf}

\hypertarget{what-new-variables-do-you-need-to-create-3-1}{%
\paragraph{3.1.2.5 What new variables do you need to create?
(3)}\label{what-new-variables-do-you-need-to-create-3-1}}

\textbf{- List the variables} The new variable large\_affected was
created above and also explained above.

\textbf{- Describe and discuss why they are needed and how you plan to
use them.} Large\_affected is needed to look at the outliers of the
individuals affected and see if there is a trend with the large values
and the states. I plan to use the logical and see if there is a
correlation with the state the breach occured in.

\hypertarget{what-type-of-covariation-occurs-between-the-variables}{%
\subsection{3.2. What type of covariation occurs between the
variables?}\label{what-type-of-covariation-occurs-between-the-variables}}

If you don't have variables of a certain type in the original dataset or
among the created variables (features), you can further create them from
the existing variables. See RDS chap.~5, 7.5 and 7.6.

\hypertarget{between-a-categorical-and-continuous-variable}{%
\subsubsection{3.2.1 Between a categorical and continuous
variable}\label{between-a-categorical-and-continuous-variable}}

\textbf{- Describe what type of visualization you can use and why.} A
boxplot of State as a categorical variable and Individuals Affected as a
continuous variable can be used. Using the box plot makes it clear the
spread of the data depending on each state of the US. Boxplots are also
compact and easier to compare the different states and the individuals
affected distributions. A bar graph can also be used to look at the
total number of individuals affected rather than the distribution by
state.

\begin{Shaded}
\begin{Highlighting}[]
\NormalTok{breaches }\SpecialCharTok{\%\textgreater{}\%}
  \FunctionTok{ggplot}\NormalTok{(}\FunctionTok{aes}\NormalTok{(}\AttributeTok{x=}\NormalTok{State, }\AttributeTok{y=}\NormalTok{Individuals\_Affected)) }\SpecialCharTok{+}
  \FunctionTok{geom\_boxplot}\NormalTok{() }\SpecialCharTok{+}
  \FunctionTok{coord\_flip}\NormalTok{() }\SpecialCharTok{+}
  \FunctionTok{labs}\NormalTok{(}\AttributeTok{title =} \StringTok{"Number of Individuals Affected in Breaches by State"}\NormalTok{)}
\end{Highlighting}
\end{Shaded}

\begin{center}\includegraphics{CourtneyKennedy_files/figure-latex/unnamed-chunk-25-1} \end{center}

\begin{Shaded}
\begin{Highlighting}[]
\NormalTok{breaches }\SpecialCharTok{\%\textgreater{}\%}
  \FunctionTok{mutate}\NormalTok{(}\AttributeTok{State =} \FunctionTok{as.factor}\NormalTok{(State) }\SpecialCharTok{\%\textgreater{}\%} \FunctionTok{fct\_infreq}\NormalTok{() }\SpecialCharTok{\%\textgreater{}\%} \FunctionTok{fct\_rev}\NormalTok{()) }\SpecialCharTok{\%\textgreater{}\%}
  \FunctionTok{ggplot}\NormalTok{(}\FunctionTok{aes}\NormalTok{(}\AttributeTok{x=}\NormalTok{State, }\AttributeTok{y=}\NormalTok{Individuals\_Affected)) }\SpecialCharTok{+}
  \FunctionTok{geom\_col}\NormalTok{() }\SpecialCharTok{+}
  \FunctionTok{coord\_flip}\NormalTok{() }\SpecialCharTok{+}
  \FunctionTok{labs}\NormalTok{(}\AttributeTok{title =} \StringTok{"Number of Individuals Affected in Breaches by State }
\StringTok{       (in order of states with most breaches)"}\NormalTok{)}
\end{Highlighting}
\end{Shaded}

\begin{center}\includegraphics{CourtneyKennedy_files/figure-latex/unnamed-chunk-25-2} \end{center}

\begin{Shaded}
\begin{Highlighting}[]
\NormalTok{total\_affected\_state }\OtherTok{\textless{}{-}}\NormalTok{ breaches }\SpecialCharTok{\%\textgreater{}\%}
  \FunctionTok{group\_by}\NormalTok{(State) }\SpecialCharTok{\%\textgreater{}\%}
  \FunctionTok{summarize}\NormalTok{(}\AttributeTok{sum\_indiv =} \FunctionTok{sum}\NormalTok{(Individuals\_Affected))}

\NormalTok{total\_affected\_state}\SpecialCharTok{$}\NormalTok{State }\OtherTok{=} \FunctionTok{with}\NormalTok{(total\_affected\_state, }\FunctionTok{reorder}\NormalTok{(State, sum\_indiv))}


\NormalTok{total\_affected\_state}\SpecialCharTok{$}\NormalTok{region }\OtherTok{\textless{}{-}} \FunctionTok{sapply}\NormalTok{(total\_affected\_state}\SpecialCharTok{$}\NormalTok{State, region\_check)}


\NormalTok{total\_affected\_state}\SpecialCharTok{$}\NormalTok{region }\OtherTok{\textless{}{-}} \FunctionTok{factor}\NormalTok{(total\_affected\_state}\SpecialCharTok{$}\NormalTok{region, }\AttributeTok{levels=}\NormalTok{ region\_levels)}

\NormalTok{total\_affected\_state }\SpecialCharTok{\%\textgreater{}\%}
  \FunctionTok{ggplot}\NormalTok{(}\FunctionTok{aes}\NormalTok{(State, sum\_indiv)) }\SpecialCharTok{+}
  \FunctionTok{geom\_col}\NormalTok{() }\SpecialCharTok{+}
  \FunctionTok{coord\_flip}\NormalTok{()}\SpecialCharTok{+}
  \FunctionTok{labs}\NormalTok{(}\AttributeTok{title =} \StringTok{"Number of Individuals Affected in Breaches by State }
\StringTok{       (in order of states with most individuals affected)"}\NormalTok{, }\AttributeTok{y=}\StringTok{"Total Number of Individuals Affected in Breaches"}\NormalTok{)}
\end{Highlighting}
\end{Shaded}

\begin{center}\includegraphics{CourtneyKennedy_files/figure-latex/unnamed-chunk-25-3} \end{center}

\textbf{- Describe the patterns and relationships you observe. Could the
identified patterns be due to coincidence (i.e.~random chance)?} The
boxplot does not appear to have a clear pattern, but there are a lot of
outliers, indicating that there were many breaches that affected a large
number of individuals. Looking at the bar graphs, the states with big
well known cities have breaches that affect the most number of people.
This could be due to chance since just one large breach could influence
a state's total individuals affected, however the trend seems to be
consistent for most of the states with big cities. The outlier in this
trend is Puerto Rico, which doesn't have a large city.

\textbf{- Describe the relationship implied by the pattern? (e.g.,
positive or negative correlation)}

There is a positive correlation between the states with big cities and
the number of individuals affected. Overall however since States is not
a measurable factor there is not a correlation. Sorting by feature may
lead to a stronger correlation by region.

\textbf{- Calculate the strength of the relationship implied by the
pattern (e.g., correlation)}

One approach at looking at correlation between categorical and
continuous variables is from
``\url{https://medium.com/@outside2SDs/an-overview-of-correlation-measures-between-categorical-and-continuous-variables-4c7f85610365}''.

The approach is to group the continuous variable using the categorical
variable, measure the variance in each group and comparing it to the
overall variance of the continuous variable. If the variance after
grouping falls down significantly, it means that the categorical
variable can explain most of the variance of the continuous variable and
so the two variables likely have a strong association. If the variables
have no correlation, then the variance in the groups is expected to be
similar to the original variance.

These calculations were done and can be seen in the data frames
state\_and\_indiv and indiv\_summary. Overall I would say there is
minimal correlation because the variance for just the indivuals affected
variable is 51920099070, and when breaches is grouped by region,
northeast, westcoast, and other variances decrease, but midwest and
south increase. Therefore grouping by region has a minimal correlation
on the Individuals Affected data.

\begin{Shaded}
\begin{Highlighting}[]
\NormalTok{state\_and\_indiv }\OtherTok{\textless{}{-}}\NormalTok{ breaches }\SpecialCharTok{\%\textgreater{}\%}
  \FunctionTok{group\_by}\NormalTok{(region) }\SpecialCharTok{\%\textgreater{}\%}
  \FunctionTok{summarize}\NormalTok{(}\AttributeTok{sd =} \FunctionTok{sd}\NormalTok{(Individuals\_Affected))}

\NormalTok{state\_and\_indiv }\SpecialCharTok{\%\textgreater{}\%}
  \FunctionTok{mutate}\NormalTok{(}\AttributeTok{var =}\NormalTok{ sd}\SpecialCharTok{\^{}}\DecValTok{2}\NormalTok{)}
\end{Highlighting}
\end{Shaded}

\begin{verbatim}
## # A tibble: 5 x 3
##   region         sd          var
## * <fct>       <dbl>        <dbl>
## 1 northeast 146987. 21605211841.
## 2 midwest   260387. 67801639077.
## 3 south     284834. 81130225641.
## 4 westcoast 150096. 22528726555.
## 5 other     106298. 11299218049.
\end{verbatim}

\begin{Shaded}
\begin{Highlighting}[]
\NormalTok{indiv\_summary }\OtherTok{\textless{}{-}}\NormalTok{ breaches }\SpecialCharTok{\%\textgreater{}\%}
  \FunctionTok{summarize}\NormalTok{(}\AttributeTok{sd =} \FunctionTok{sd}\NormalTok{(Individuals\_Affected))}

\NormalTok{indiv\_summary }\SpecialCharTok{\%\textgreater{}\%}
  \FunctionTok{mutate}\NormalTok{(}\AttributeTok{var =}\NormalTok{ sd}\SpecialCharTok{\^{}}\DecValTok{2}\NormalTok{)}
\end{Highlighting}
\end{Shaded}

\begin{verbatim}
## # A tibble: 1 x 2
##        sd          var
##     <dbl>        <dbl>
## 1 227860. 51920099070.
\end{verbatim}

\textbf{- Discuss what other variables might affect the relationship}
Some other variables that may affect the relationship between State and
Individuals Affected are type, length of breach, year, and location of
breached information, all of which are being explored by other members.

\textbf{- Does the relationship change if you look at individual
subgroups of the data? Please discuss and demonstrate.}

Looking at whether the breach was below 20,000 people affected or above,
gives new insight into the relationship between State and Individuals
affected. In the small breaches, WY and LA and PR all stand out with a
box plot that has a median higher than the other states. In just the
large breaches VA, GA, FL, IL, MA, CA and TN are all positively skewed
in the large breach, meaning they have breaches with a larger variation
above the median.

\begin{Shaded}
\begin{Highlighting}[]
\NormalTok{small\_breach }\OtherTok{\textless{}{-}}\NormalTok{ breaches }\SpecialCharTok{\%\textgreater{}\%}
  \FunctionTok{filter}\NormalTok{(large\_affected }\SpecialCharTok{==} \ConstantTok{FALSE}\NormalTok{) }\SpecialCharTok{\%\textgreater{}\%}
  \FunctionTok{ggplot}\NormalTok{(}\FunctionTok{aes}\NormalTok{(}\AttributeTok{x=}\NormalTok{State, }\AttributeTok{y=}\NormalTok{Individuals\_Affected)) }\SpecialCharTok{+}
  \FunctionTok{geom\_boxplot}\NormalTok{() }\SpecialCharTok{+}
  \FunctionTok{coord\_flip}\NormalTok{() }\SpecialCharTok{+}
  \FunctionTok{labs}\NormalTok{(}\AttributeTok{title =} \StringTok{"Small Breach: Less than 20,000 Affected"}\NormalTok{)}

\NormalTok{large\_breach }\OtherTok{\textless{}{-}}\NormalTok{ breaches }\SpecialCharTok{\%\textgreater{}\%}
  \FunctionTok{filter}\NormalTok{(large\_affected }\SpecialCharTok{==} \ConstantTok{TRUE}\NormalTok{) }\SpecialCharTok{\%\textgreater{}\%}
  \FunctionTok{ggplot}\NormalTok{(}\FunctionTok{aes}\NormalTok{(}\AttributeTok{x=}\NormalTok{State, }\AttributeTok{y=}\NormalTok{Individuals\_Affected)) }\SpecialCharTok{+}
  \FunctionTok{geom\_boxplot}\NormalTok{()}\SpecialCharTok{+}
  \FunctionTok{coord\_flip}\NormalTok{() }\SpecialCharTok{+}
  \FunctionTok{labs}\NormalTok{(}\AttributeTok{title =} \StringTok{"Large Breach: More than 20,000 Affected"}\NormalTok{)}

\NormalTok{small\_breach}
\end{Highlighting}
\end{Shaded}

\begin{center}\includegraphics{CourtneyKennedy_files/figure-latex/unnamed-chunk-27-1} \end{center}

\begin{Shaded}
\begin{Highlighting}[]
\NormalTok{large\_breach}
\end{Highlighting}
\end{Shaded}

\begin{center}\includegraphics{CourtneyKennedy_files/figure-latex/unnamed-chunk-27-2} \end{center}

\textbf{- Demonstrate if converting the type of these variables help
exploring the relationship.} Converting the State to a factor, called
region we are able to explore how the distribution and total number of
individuals affected changes by region.

\begin{Shaded}
\begin{Highlighting}[]
\NormalTok{small\_breach\_region }\OtherTok{\textless{}{-}}\NormalTok{ breaches }\SpecialCharTok{\%\textgreater{}\%}
  \FunctionTok{filter}\NormalTok{(large\_affected }\SpecialCharTok{==} \ConstantTok{FALSE}\NormalTok{) }\SpecialCharTok{\%\textgreater{}\%}
  \FunctionTok{ggplot}\NormalTok{(}\FunctionTok{aes}\NormalTok{(}\AttributeTok{x=}\NormalTok{region, }\AttributeTok{y=}\NormalTok{Individuals\_Affected)) }\SpecialCharTok{+}
  \FunctionTok{geom\_boxplot}\NormalTok{() }\SpecialCharTok{+}
  \FunctionTok{coord\_flip}\NormalTok{() }\SpecialCharTok{+}
  \FunctionTok{labs}\NormalTok{(}\AttributeTok{title =} \StringTok{"Small Breach: Less than 20,000 Affected"}\NormalTok{)}

\NormalTok{large\_breach\_region }\OtherTok{\textless{}{-}}\NormalTok{ breaches }\SpecialCharTok{\%\textgreater{}\%}
  \FunctionTok{filter}\NormalTok{(large\_affected }\SpecialCharTok{==} \ConstantTok{TRUE}\NormalTok{) }\SpecialCharTok{\%\textgreater{}\%}
  \FunctionTok{ggplot}\NormalTok{(}\FunctionTok{aes}\NormalTok{(}\AttributeTok{x=}\NormalTok{region, }\AttributeTok{y=}\NormalTok{Individuals\_Affected)) }\SpecialCharTok{+}
  \FunctionTok{geom\_boxplot}\NormalTok{()}\SpecialCharTok{+}
  \FunctionTok{coord\_flip}\NormalTok{() }\SpecialCharTok{+}
  \FunctionTok{labs}\NormalTok{(}\AttributeTok{title =} \StringTok{"Large Breach: More than 20,000 Affected"}\NormalTok{)}


\NormalTok{region\_bar }\OtherTok{\textless{}{-}}\NormalTok{ total\_affected\_state }\SpecialCharTok{\%\textgreater{}\%}
  \FunctionTok{ggplot}\NormalTok{(}\FunctionTok{aes}\NormalTok{(}\AttributeTok{x=}\NormalTok{region, }\AttributeTok{y=}\NormalTok{sum\_indiv)) }\SpecialCharTok{+}
  \FunctionTok{geom\_col}\NormalTok{()}\SpecialCharTok{+}
  \FunctionTok{coord\_flip}\NormalTok{() }\SpecialCharTok{+}
  \FunctionTok{labs}\NormalTok{(}\AttributeTok{title =} \StringTok{"Bar"}\NormalTok{)}

\FunctionTok{ggarrange}\NormalTok{(small\_breach\_region, large\_breach\_region, }\AttributeTok{nrow =} \DecValTok{2}\NormalTok{)}
\end{Highlighting}
\end{Shaded}

\includegraphics{CourtneyKennedy_files/figure-latex/unnamed-chunk-28-1.pdf}

\begin{Shaded}
\begin{Highlighting}[]
\NormalTok{region\_bar}
\end{Highlighting}
\end{Shaded}

\includegraphics{CourtneyKennedy_files/figure-latex/unnamed-chunk-28-2.pdf}

In small breaches, the other states have a higher median value than the
other regions, however the total number of individuals affected is the
lowest. The south's distribution has the most variation in the large
breaches and also has the highest total number of individuals affected.

\begin{Shaded}
\begin{Highlighting}[]
\NormalTok{northeast\_states }\OtherTok{\textless{}{-}}\NormalTok{ breaches }\SpecialCharTok{\%\textgreater{}\%}
  \FunctionTok{filter}\NormalTok{(northeast }\SpecialCharTok{==} \ConstantTok{TRUE}\NormalTok{) }\SpecialCharTok{\%\textgreater{}\%}
  \FunctionTok{ggplot}\NormalTok{(}\FunctionTok{aes}\NormalTok{(}\AttributeTok{x=}\NormalTok{State, }\AttributeTok{y=}\NormalTok{Individuals\_Affected)) }\SpecialCharTok{+}
  \FunctionTok{geom\_boxplot}\NormalTok{() }\SpecialCharTok{+} 
  \FunctionTok{coord\_flip}\NormalTok{() }\SpecialCharTok{+}
  \FunctionTok{labs}\NormalTok{(}\AttributeTok{title =} \StringTok{"All Breaches in Northeast"}\NormalTok{)}

\NormalTok{large\_northeast\_states }\OtherTok{\textless{}{-}}\NormalTok{ breaches }\SpecialCharTok{\%\textgreater{}\%}
  \FunctionTok{filter}\NormalTok{(northeast }\SpecialCharTok{==} \ConstantTok{TRUE}\NormalTok{, large\_affected }\SpecialCharTok{==} \ConstantTok{TRUE}\NormalTok{ ) }\SpecialCharTok{\%\textgreater{}\%}
  \FunctionTok{ggplot}\NormalTok{(}\FunctionTok{aes}\NormalTok{(}\AttributeTok{x=}\NormalTok{State, }\AttributeTok{y=}\NormalTok{Individuals\_Affected)) }\SpecialCharTok{+}
  \FunctionTok{geom\_boxplot}\NormalTok{() }\SpecialCharTok{+} 
  \FunctionTok{coord\_flip}\NormalTok{() }\SpecialCharTok{+}
  \FunctionTok{labs}\NormalTok{(}\AttributeTok{title =} \StringTok{"Large Breaches in Northeast"}\NormalTok{)}

\NormalTok{small\_northeast\_states }\OtherTok{\textless{}{-}}\NormalTok{ breaches }\SpecialCharTok{\%\textgreater{}\%}
  \FunctionTok{filter}\NormalTok{(northeast }\SpecialCharTok{==} \ConstantTok{TRUE}\NormalTok{, large\_affected }\SpecialCharTok{==} \ConstantTok{FALSE}\NormalTok{ ) }\SpecialCharTok{\%\textgreater{}\%}
  \FunctionTok{ggplot}\NormalTok{(}\FunctionTok{aes}\NormalTok{(}\AttributeTok{x=}\NormalTok{State, }\AttributeTok{y=}\NormalTok{Individuals\_Affected)) }\SpecialCharTok{+}
  \FunctionTok{geom\_boxplot}\NormalTok{() }\SpecialCharTok{+} 
  \FunctionTok{coord\_flip}\NormalTok{() }\SpecialCharTok{+}
  \FunctionTok{labs}\NormalTok{(}\AttributeTok{title =} \StringTok{"Small Breaches in Northeast"}\NormalTok{)}

\NormalTok{northeast\_bar }\OtherTok{\textless{}{-}}\NormalTok{ total\_affected\_state }\SpecialCharTok{\%\textgreater{}\%}
  \FunctionTok{filter}\NormalTok{(region }\SpecialCharTok{==} \StringTok{"northeast"}\NormalTok{) }\SpecialCharTok{\%\textgreater{}\%}
  \FunctionTok{ggplot}\NormalTok{(}\FunctionTok{aes}\NormalTok{(}\AttributeTok{x=}\NormalTok{State, }\AttributeTok{y=}\NormalTok{sum\_indiv)) }\SpecialCharTok{+}
  \FunctionTok{geom\_col}\NormalTok{()}\SpecialCharTok{+}
  \FunctionTok{coord\_flip}\NormalTok{() }\SpecialCharTok{+}
  \FunctionTok{labs}\NormalTok{(}\AttributeTok{title =} \StringTok{"Bar"}\NormalTok{)}

\NormalTok{northeast\_states}
\end{Highlighting}
\end{Shaded}

\includegraphics{CourtneyKennedy_files/figure-latex/unnamed-chunk-29-1.pdf}

\begin{Shaded}
\begin{Highlighting}[]
\NormalTok{large\_northeast\_states}
\end{Highlighting}
\end{Shaded}

\includegraphics{CourtneyKennedy_files/figure-latex/unnamed-chunk-29-2.pdf}

\begin{Shaded}
\begin{Highlighting}[]
\NormalTok{small\_northeast\_states}
\end{Highlighting}
\end{Shaded}

\includegraphics{CourtneyKennedy_files/figure-latex/unnamed-chunk-29-3.pdf}

\begin{Shaded}
\begin{Highlighting}[]
\NormalTok{northeast\_bar}
\end{Highlighting}
\end{Shaded}

\includegraphics{CourtneyKennedy_files/figure-latex/unnamed-chunk-29-4.pdf}
New Jersey has the highest median of individuals affected for large
breaches, but does not stand out in small breaches. New York does not
stand out in either small or large breaches, however it does have an
extremely large outlier value, which makes it the largest total number
of individuals affected.

\begin{Shaded}
\begin{Highlighting}[]
\NormalTok{westcoast\_states }\OtherTok{\textless{}{-}}\NormalTok{ breaches }\SpecialCharTok{\%\textgreater{}\%}
  \FunctionTok{filter}\NormalTok{(westcoast }\SpecialCharTok{==} \ConstantTok{TRUE}\NormalTok{) }\SpecialCharTok{\%\textgreater{}\%}
  \FunctionTok{ggplot}\NormalTok{(}\FunctionTok{aes}\NormalTok{(}\AttributeTok{x=}\NormalTok{State, }\AttributeTok{y=}\NormalTok{Individuals\_Affected)) }\SpecialCharTok{+}
  \FunctionTok{geom\_boxplot}\NormalTok{() }\SpecialCharTok{+} 
  \FunctionTok{coord\_flip}\NormalTok{() }\SpecialCharTok{+}
  \FunctionTok{labs}\NormalTok{(}\AttributeTok{title =} \StringTok{"All Breaches in Westcoast"}\NormalTok{)}

\NormalTok{large\_westcoast\_states }\OtherTok{\textless{}{-}}\NormalTok{ breaches }\SpecialCharTok{\%\textgreater{}\%}
  \FunctionTok{filter}\NormalTok{(westcoast }\SpecialCharTok{==} \ConstantTok{TRUE}\NormalTok{, large\_affected }\SpecialCharTok{==} \ConstantTok{TRUE}\NormalTok{ ) }\SpecialCharTok{\%\textgreater{}\%}
  \FunctionTok{ggplot}\NormalTok{(}\FunctionTok{aes}\NormalTok{(}\AttributeTok{x=}\NormalTok{State, }\AttributeTok{y=}\NormalTok{Individuals\_Affected)) }\SpecialCharTok{+}
  \FunctionTok{geom\_boxplot}\NormalTok{() }\SpecialCharTok{+} 
  \FunctionTok{coord\_flip}\NormalTok{() }\SpecialCharTok{+}
  \FunctionTok{labs}\NormalTok{(}\AttributeTok{title =} \StringTok{"Large Breaches in Westcoast"}\NormalTok{)}

\NormalTok{small\_westcoast\_states }\OtherTok{\textless{}{-}}\NormalTok{ breaches }\SpecialCharTok{\%\textgreater{}\%}
  \FunctionTok{filter}\NormalTok{(westcoast }\SpecialCharTok{==} \ConstantTok{TRUE}\NormalTok{, large\_affected }\SpecialCharTok{==} \ConstantTok{FALSE}\NormalTok{ ) }\SpecialCharTok{\%\textgreater{}\%}
  \FunctionTok{ggplot}\NormalTok{(}\FunctionTok{aes}\NormalTok{(}\AttributeTok{x=}\NormalTok{State, }\AttributeTok{y=}\NormalTok{Individuals\_Affected)) }\SpecialCharTok{+}
  \FunctionTok{geom\_boxplot}\NormalTok{() }\SpecialCharTok{+} 
  \FunctionTok{coord\_flip}\NormalTok{() }\SpecialCharTok{+}
  \FunctionTok{labs}\NormalTok{(}\AttributeTok{title =} \StringTok{"Small Breaches in Westcoast"}\NormalTok{)}

\NormalTok{westcoast\_bar }\OtherTok{\textless{}{-}}\NormalTok{ total\_affected\_state }\SpecialCharTok{\%\textgreater{}\%}
  \FunctionTok{filter}\NormalTok{(region }\SpecialCharTok{==} \StringTok{"westcoast"}\NormalTok{) }\SpecialCharTok{\%\textgreater{}\%}
  \FunctionTok{ggplot}\NormalTok{(}\FunctionTok{aes}\NormalTok{(}\AttributeTok{x=}\NormalTok{State, }\AttributeTok{y=}\NormalTok{sum\_indiv)) }\SpecialCharTok{+}
  \FunctionTok{geom\_col}\NormalTok{()}\SpecialCharTok{+}
  \FunctionTok{coord\_flip}\NormalTok{() }\SpecialCharTok{+}
  \FunctionTok{labs}\NormalTok{(}\AttributeTok{title =} \StringTok{"Total Indivduals Affected by State in Westcoast"}\NormalTok{)}

\NormalTok{westcoast\_states}
\end{Highlighting}
\end{Shaded}

\includegraphics{CourtneyKennedy_files/figure-latex/unnamed-chunk-30-1.pdf}

\begin{Shaded}
\begin{Highlighting}[]
\NormalTok{large\_westcoast\_states}
\end{Highlighting}
\end{Shaded}

\includegraphics{CourtneyKennedy_files/figure-latex/unnamed-chunk-30-2.pdf}

\begin{Shaded}
\begin{Highlighting}[]
\NormalTok{small\_westcoast\_states}
\end{Highlighting}
\end{Shaded}

\includegraphics{CourtneyKennedy_files/figure-latex/unnamed-chunk-30-3.pdf}

\begin{Shaded}
\begin{Highlighting}[]
\NormalTok{westcoast\_bar}
\end{Highlighting}
\end{Shaded}

\includegraphics{CourtneyKennedy_files/figure-latex/unnamed-chunk-30-4.pdf}
There are not many large breaches on the westcoast, with CA as the
exception, having 3 outlier breaches that lead to CA having the highest
total number of individuals affected on the Westcoast. Overall the small
breaches have a median of slightly below 2500 indivuals affected, with
ID and WY standing out and having a higher median. However both ID and
WY are two of the lowest total number of individuals affected.

\begin{Shaded}
\begin{Highlighting}[]
\NormalTok{midwest\_states }\OtherTok{\textless{}{-}}\NormalTok{ breaches }\SpecialCharTok{\%\textgreater{}\%}
  \FunctionTok{filter}\NormalTok{(midwest }\SpecialCharTok{==} \ConstantTok{TRUE}\NormalTok{) }\SpecialCharTok{\%\textgreater{}\%}
  \FunctionTok{ggplot}\NormalTok{(}\FunctionTok{aes}\NormalTok{(}\AttributeTok{x=}\NormalTok{State, }\AttributeTok{y=}\NormalTok{Individuals\_Affected)) }\SpecialCharTok{+}
  \FunctionTok{geom\_boxplot}\NormalTok{() }\SpecialCharTok{+} 
  \FunctionTok{coord\_flip}\NormalTok{() }\SpecialCharTok{+}
  \FunctionTok{labs}\NormalTok{(}\AttributeTok{title =} \StringTok{"All Breaches in midwest"}\NormalTok{)}

\NormalTok{large\_midwest\_states }\OtherTok{\textless{}{-}}\NormalTok{ breaches }\SpecialCharTok{\%\textgreater{}\%}
  \FunctionTok{filter}\NormalTok{(midwest }\SpecialCharTok{==} \ConstantTok{TRUE}\NormalTok{, large\_affected }\SpecialCharTok{==} \ConstantTok{TRUE}\NormalTok{ ) }\SpecialCharTok{\%\textgreater{}\%}
  \FunctionTok{ggplot}\NormalTok{(}\FunctionTok{aes}\NormalTok{(}\AttributeTok{x=}\NormalTok{State, }\AttributeTok{y=}\NormalTok{Individuals\_Affected)) }\SpecialCharTok{+}
  \FunctionTok{geom\_boxplot}\NormalTok{() }\SpecialCharTok{+} 
  \FunctionTok{coord\_flip}\NormalTok{() }\SpecialCharTok{+}
  \FunctionTok{labs}\NormalTok{(}\AttributeTok{title =} \StringTok{"Large Breaches in midwest"}\NormalTok{)}

\NormalTok{small\_midwest\_states }\OtherTok{\textless{}{-}}\NormalTok{ breaches }\SpecialCharTok{\%\textgreater{}\%}
  \FunctionTok{filter}\NormalTok{(midwest }\SpecialCharTok{==} \ConstantTok{TRUE}\NormalTok{, large\_affected }\SpecialCharTok{==} \ConstantTok{FALSE}\NormalTok{ ) }\SpecialCharTok{\%\textgreater{}\%}
  \FunctionTok{ggplot}\NormalTok{(}\FunctionTok{aes}\NormalTok{(}\AttributeTok{x=}\NormalTok{State, }\AttributeTok{y=}\NormalTok{Individuals\_Affected)) }\SpecialCharTok{+}
  \FunctionTok{geom\_boxplot}\NormalTok{() }\SpecialCharTok{+} 
  \FunctionTok{coord\_flip}\NormalTok{() }\SpecialCharTok{+}
  \FunctionTok{labs}\NormalTok{(}\AttributeTok{title =} \StringTok{"Small Breaches in midwest"}\NormalTok{)}

\NormalTok{midwest\_bar }\OtherTok{\textless{}{-}}\NormalTok{ total\_affected\_state }\SpecialCharTok{\%\textgreater{}\%}
  \FunctionTok{filter}\NormalTok{(region }\SpecialCharTok{==} \StringTok{"midwest"}\NormalTok{) }\SpecialCharTok{\%\textgreater{}\%}
  \FunctionTok{ggplot}\NormalTok{(}\FunctionTok{aes}\NormalTok{(}\AttributeTok{x=}\NormalTok{State, }\AttributeTok{y=}\NormalTok{sum\_indiv)) }\SpecialCharTok{+}
  \FunctionTok{geom\_col}\NormalTok{()}\SpecialCharTok{+}
  \FunctionTok{coord\_flip}\NormalTok{() }\SpecialCharTok{+}
  \FunctionTok{labs}\NormalTok{(}\AttributeTok{title =} \StringTok{"Total Indivduals Affected by State in midwest"}\NormalTok{)}

\NormalTok{midwest\_states}
\end{Highlighting}
\end{Shaded}

\includegraphics{CourtneyKennedy_files/figure-latex/unnamed-chunk-31-1.pdf}

\begin{Shaded}
\begin{Highlighting}[]
\NormalTok{large\_midwest\_states}
\end{Highlighting}
\end{Shaded}

\includegraphics{CourtneyKennedy_files/figure-latex/unnamed-chunk-31-2.pdf}

\begin{Shaded}
\begin{Highlighting}[]
\NormalTok{small\_midwest\_states}
\end{Highlighting}
\end{Shaded}

\includegraphics{CourtneyKennedy_files/figure-latex/unnamed-chunk-31-3.pdf}

\begin{Shaded}
\begin{Highlighting}[]
\NormalTok{midwest\_bar}
\end{Highlighting}
\end{Shaded}

\includegraphics{CourtneyKennedy_files/figure-latex/unnamed-chunk-31-4.pdf}
In the small breaches, there are many outlier values, but no states
median is significantly higher than any of the others. SD is the
exception, but there is only one small breach and that value is
therefore the median. IL stands out in the large breaches, having the
widest distribution and the largest outlier value. IL is significantly
higher in the total number of individuals affected.

\begin{Shaded}
\begin{Highlighting}[]
\NormalTok{south\_states }\OtherTok{\textless{}{-}}\NormalTok{ breaches }\SpecialCharTok{\%\textgreater{}\%}
  \FunctionTok{filter}\NormalTok{(south }\SpecialCharTok{==} \ConstantTok{TRUE}\NormalTok{) }\SpecialCharTok{\%\textgreater{}\%}
  \FunctionTok{ggplot}\NormalTok{(}\FunctionTok{aes}\NormalTok{(}\AttributeTok{x=}\NormalTok{State, }\AttributeTok{y=}\NormalTok{Individuals\_Affected)) }\SpecialCharTok{+}
  \FunctionTok{geom\_boxplot}\NormalTok{() }\SpecialCharTok{+} 
  \FunctionTok{coord\_flip}\NormalTok{() }\SpecialCharTok{+}
  \FunctionTok{labs}\NormalTok{(}\AttributeTok{title =} \StringTok{"All Breaches in south"}\NormalTok{)}

\NormalTok{large\_south\_states }\OtherTok{\textless{}{-}}\NormalTok{ breaches }\SpecialCharTok{\%\textgreater{}\%}
  \FunctionTok{filter}\NormalTok{(south }\SpecialCharTok{==} \ConstantTok{TRUE}\NormalTok{, large\_affected }\SpecialCharTok{==} \ConstantTok{TRUE}\NormalTok{ ) }\SpecialCharTok{\%\textgreater{}\%}
  \FunctionTok{ggplot}\NormalTok{(}\FunctionTok{aes}\NormalTok{(}\AttributeTok{x=}\NormalTok{State, }\AttributeTok{y=}\NormalTok{Individuals\_Affected)) }\SpecialCharTok{+}
  \FunctionTok{geom\_boxplot}\NormalTok{() }\SpecialCharTok{+} 
  \FunctionTok{coord\_flip}\NormalTok{() }\SpecialCharTok{+}
  \FunctionTok{labs}\NormalTok{(}\AttributeTok{title =} \StringTok{"Large Breaches in south"}\NormalTok{)}

\NormalTok{small\_south\_states }\OtherTok{\textless{}{-}}\NormalTok{ breaches }\SpecialCharTok{\%\textgreater{}\%}
  \FunctionTok{filter}\NormalTok{(south }\SpecialCharTok{==} \ConstantTok{TRUE}\NormalTok{, large\_affected }\SpecialCharTok{==} \ConstantTok{FALSE}\NormalTok{ ) }\SpecialCharTok{\%\textgreater{}\%}
  \FunctionTok{ggplot}\NormalTok{(}\FunctionTok{aes}\NormalTok{(}\AttributeTok{x=}\NormalTok{State, }\AttributeTok{y=}\NormalTok{Individuals\_Affected)) }\SpecialCharTok{+}
  \FunctionTok{geom\_boxplot}\NormalTok{() }\SpecialCharTok{+} 
  \FunctionTok{coord\_flip}\NormalTok{() }\SpecialCharTok{+}
  \FunctionTok{labs}\NormalTok{(}\AttributeTok{title =} \StringTok{"Small Breaches in south"}\NormalTok{)}

\NormalTok{south\_bar }\OtherTok{\textless{}{-}}\NormalTok{ total\_affected\_state }\SpecialCharTok{\%\textgreater{}\%}
  \FunctionTok{filter}\NormalTok{(region }\SpecialCharTok{==} \StringTok{"south"}\NormalTok{) }\SpecialCharTok{\%\textgreater{}\%}
  \FunctionTok{ggplot}\NormalTok{(}\FunctionTok{aes}\NormalTok{(}\AttributeTok{x=}\NormalTok{State, }\AttributeTok{y=}\NormalTok{sum\_indiv)) }\SpecialCharTok{+}
  \FunctionTok{geom\_col}\NormalTok{()}\SpecialCharTok{+}
  \FunctionTok{coord\_flip}\NormalTok{() }\SpecialCharTok{+}
  \FunctionTok{labs}\NormalTok{(}\AttributeTok{title =} \StringTok{"Total Indivduals Affected by State in south"}\NormalTok{)}

\NormalTok{south\_states}
\end{Highlighting}
\end{Shaded}

\includegraphics{CourtneyKennedy_files/figure-latex/unnamed-chunk-32-1.pdf}

\begin{Shaded}
\begin{Highlighting}[]
\NormalTok{large\_south\_states}
\end{Highlighting}
\end{Shaded}

\includegraphics{CourtneyKennedy_files/figure-latex/unnamed-chunk-32-2.pdf}

\begin{Shaded}
\begin{Highlighting}[]
\NormalTok{small\_south\_states}
\end{Highlighting}
\end{Shaded}

\includegraphics{CourtneyKennedy_files/figure-latex/unnamed-chunk-32-3.pdf}

\begin{Shaded}
\begin{Highlighting}[]
\NormalTok{south\_bar}
\end{Highlighting}
\end{Shaded}

\includegraphics{CourtneyKennedy_files/figure-latex/unnamed-chunk-32-4.pdf}
In the south the top 5 states by total individuals affected are VA, FL,
TN, TX, and AL, all of which have distribution that are more spread out
in large breaches, other than TX. Texas does have 2 outlier values in
the large breaches that bring the total individuals affected up. In
smaller breaches, LA has a higher median, but the total number of
individuals affected is one of the lowest in the south.

\begin{Shaded}
\begin{Highlighting}[]
\NormalTok{other\_states }\OtherTok{\textless{}{-}}\NormalTok{ breaches }\SpecialCharTok{\%\textgreater{}\%}
  \FunctionTok{filter}\NormalTok{(region }\SpecialCharTok{==} \StringTok{"other"}\NormalTok{) }\SpecialCharTok{\%\textgreater{}\%}
  \FunctionTok{ggplot}\NormalTok{(}\FunctionTok{aes}\NormalTok{(}\AttributeTok{x=}\NormalTok{State, }\AttributeTok{y=}\NormalTok{Individuals\_Affected)) }\SpecialCharTok{+}
  \FunctionTok{geom\_boxplot}\NormalTok{() }\SpecialCharTok{+} 
  \FunctionTok{coord\_flip}\NormalTok{() }\SpecialCharTok{+}
  \FunctionTok{labs}\NormalTok{(}\AttributeTok{title =} \StringTok{"All Breaches in other states"}\NormalTok{)}

\NormalTok{large\_other\_states }\OtherTok{\textless{}{-}}\NormalTok{ breaches }\SpecialCharTok{\%\textgreater{}\%}
  \FunctionTok{filter}\NormalTok{(region }\SpecialCharTok{==} \StringTok{"other"}\NormalTok{, large\_affected }\SpecialCharTok{==} \ConstantTok{TRUE}\NormalTok{ ) }\SpecialCharTok{\%\textgreater{}\%}
  \FunctionTok{ggplot}\NormalTok{(}\FunctionTok{aes}\NormalTok{(}\AttributeTok{x=}\NormalTok{State, }\AttributeTok{y=}\NormalTok{Individuals\_Affected)) }\SpecialCharTok{+}
  \FunctionTok{geom\_boxplot}\NormalTok{() }\SpecialCharTok{+} 
  \FunctionTok{coord\_flip}\NormalTok{() }\SpecialCharTok{+}
  \FunctionTok{labs}\NormalTok{(}\AttributeTok{title =} \StringTok{"Large Breaches in other states"}\NormalTok{)}

\NormalTok{small\_other\_states }\OtherTok{\textless{}{-}}\NormalTok{ breaches }\SpecialCharTok{\%\textgreater{}\%}
  \FunctionTok{filter}\NormalTok{(region }\SpecialCharTok{==} \StringTok{"other"}\NormalTok{, large\_affected }\SpecialCharTok{==} \ConstantTok{FALSE}\NormalTok{ ) }\SpecialCharTok{\%\textgreater{}\%}
  \FunctionTok{ggplot}\NormalTok{(}\FunctionTok{aes}\NormalTok{(}\AttributeTok{x=}\NormalTok{State, }\AttributeTok{y=}\NormalTok{Individuals\_Affected)) }\SpecialCharTok{+}
  \FunctionTok{geom\_boxplot}\NormalTok{() }\SpecialCharTok{+} 
  \FunctionTok{coord\_flip}\NormalTok{() }\SpecialCharTok{+}
  \FunctionTok{labs}\NormalTok{(}\AttributeTok{title =} \StringTok{"Small Breaches in other states"}\NormalTok{)}

\NormalTok{other\_bar }\OtherTok{\textless{}{-}}\NormalTok{ total\_affected\_state }\SpecialCharTok{\%\textgreater{}\%}
  \FunctionTok{filter}\NormalTok{(region }\SpecialCharTok{==} \StringTok{"other"}\NormalTok{) }\SpecialCharTok{\%\textgreater{}\%}
  \FunctionTok{ggplot}\NormalTok{(}\FunctionTok{aes}\NormalTok{(}\AttributeTok{x=}\NormalTok{State, }\AttributeTok{y=}\NormalTok{sum\_indiv)) }\SpecialCharTok{+}
  \FunctionTok{geom\_col}\NormalTok{()}\SpecialCharTok{+}
  \FunctionTok{coord\_flip}\NormalTok{() }\SpecialCharTok{+}
  \FunctionTok{labs}\NormalTok{(}\AttributeTok{title =} \StringTok{"Total Indivduals Affected by State in other states"}\NormalTok{)}

\NormalTok{other\_states}
\end{Highlighting}
\end{Shaded}

\includegraphics{CourtneyKennedy_files/figure-latex/unnamed-chunk-33-1.pdf}

\begin{Shaded}
\begin{Highlighting}[]
\NormalTok{large\_other\_states}
\end{Highlighting}
\end{Shaded}

\includegraphics{CourtneyKennedy_files/figure-latex/unnamed-chunk-33-2.pdf}

\begin{Shaded}
\begin{Highlighting}[]
\NormalTok{small\_other\_states}
\end{Highlighting}
\end{Shaded}

\includegraphics{CourtneyKennedy_files/figure-latex/unnamed-chunk-33-3.pdf}

\begin{Shaded}
\begin{Highlighting}[]
\NormalTok{other\_bar}
\end{Highlighting}
\end{Shaded}

\includegraphics{CourtneyKennedy_files/figure-latex/unnamed-chunk-33-4.pdf}
The only state that truly plays a role in breaches in other is PR, which
has some outlier individuals affected in the large breach, that brings
the total number of individuals up. HI only had 1 breach and it was a
small breach.

\textbf{Normalized individuals} Since each region has different total
populations, looking at the normalized Individuals Affected gives a
better picture of how impactful the breaches were. Also the other region
only includes PR and HI so any breach

\begin{Shaded}
\begin{Highlighting}[]
\NormalTok{state\_populations }\OtherTok{\textless{}{-}} \FunctionTok{read\_csv}\NormalTok{(}\StringTok{"state{-}population.csv"}\NormalTok{)}
\end{Highlighting}
\end{Shaded}

\begin{verbatim}
## Warning: Missing column names filled in: 'X3' [3], 'X4' [4]
\end{verbatim}

\begin{verbatim}
## 
## -- Column specification --------------------------------------------------------
## cols(
##   State = col_character(),
##   Population = col_number(),
##   X3 = col_logical(),
##   X4 = col_logical()
## )
\end{verbatim}

\begin{Shaded}
\begin{Highlighting}[]
\NormalTok{state\_populations }\OtherTok{\textless{}{-}}\NormalTok{ state\_populations }\SpecialCharTok{\%\textgreater{}\%} \FunctionTok{select}\NormalTok{(State, Population)}
\end{Highlighting}
\end{Shaded}

\begin{Shaded}
\begin{Highlighting}[]
\NormalTok{total\_affected\_state }\OtherTok{\textless{}{-}}\NormalTok{ total\_affected\_state }\SpecialCharTok{\%\textgreater{}\%}
  \FunctionTok{left\_join}\NormalTok{(state\_populations)}
\end{Highlighting}
\end{Shaded}

\begin{verbatim}
## Joining, by = "State"
\end{verbatim}

\begin{Shaded}
\begin{Highlighting}[]
\NormalTok{total\_affected\_state }\OtherTok{\textless{}{-}}\NormalTok{ total\_affected\_state }\SpecialCharTok{\%\textgreater{}\%}
  \FunctionTok{mutate}\NormalTok{(}\AttributeTok{normalized\_state\_indiv =}\NormalTok{ sum\_indiv }\SpecialCharTok{/}\NormalTok{ Population)}

\NormalTok{total\_affected\_state\_sorted }\OtherTok{\textless{}{-}}\NormalTok{ total\_affected\_state[}\FunctionTok{order}\NormalTok{(total\_affected\_state}\SpecialCharTok{$}\NormalTok{normalized\_state\_indiv),]}
\end{Highlighting}
\end{Shaded}

\begin{Shaded}
\begin{Highlighting}[]
\NormalTok{total\_affected\_state\_sorted}\SpecialCharTok{$}\NormalTok{State }\OtherTok{\textless{}{-}} \FunctionTok{factor}\NormalTok{(total\_affected\_state\_sorted}\SpecialCharTok{$}\NormalTok{State, }\AttributeTok{levels =}\NormalTok{ total\_affected\_state\_sorted}\SpecialCharTok{$}\NormalTok{State)}
\NormalTok{total\_affected\_state\_sorted }\SpecialCharTok{\%\textgreater{}\%}
  \FunctionTok{ggplot}\NormalTok{(}\FunctionTok{aes}\NormalTok{(State, normalized\_state\_indiv)) }\SpecialCharTok{+} 
  \FunctionTok{geom\_col}\NormalTok{() }\SpecialCharTok{+} 
  \FunctionTok{coord\_flip}\NormalTok{() }\SpecialCharTok{+}
  \FunctionTok{ylab}\NormalTok{(}\StringTok{"Proportion of Region Affected by Breaches"}\NormalTok{)}
\end{Highlighting}
\end{Shaded}

\begin{center}\includegraphics{CourtneyKennedy_files/figure-latex/unnamed-chunk-36-1} \end{center}

By normalizing the individuals affected by the state population a better
comparison of the breaches affects in a state. Virginia, Puerto Rico,
and Illinois are the states with the top percentage of individuals
affected in their state. Therefore in these states getting affected by a
breach is more likely.

\begin{Shaded}
\begin{Highlighting}[]
\NormalTok{northeast\_total\_population }\OtherTok{\textless{}{-}} \DecValTok{56059240}
\NormalTok{midwest\_total\_population }\OtherTok{\textless{}{-}} \DecValTok{68126781}
\NormalTok{west\_total\_population }\OtherTok{\textless{}{-}} \DecValTok{77257329}
\NormalTok{south\_total\_population }\OtherTok{\textless{}{-}} \DecValTok{123542189}
\NormalTok{other\_total\_population }\OtherTok{\textless{}{-}} \DecValTok{3375000} \SpecialCharTok{+} \DecValTok{1424000}

\NormalTok{normal\_function }\OtherTok{\textless{}{-}} \ControlFlowTok{function}\NormalTok{(x) \{}
  \ControlFlowTok{if}\NormalTok{(}\FunctionTok{is.na}\NormalTok{(x))\{}
    \FunctionTok{return}\NormalTok{(}\ConstantTok{NA}\NormalTok{)}
\NormalTok{  \}}
  \ControlFlowTok{else} \ControlFlowTok{if}\NormalTok{(x }\SpecialCharTok{==} \StringTok{"northeast"}\NormalTok{)\{}
    \FunctionTok{return}\NormalTok{(northeast\_total\_population)}
\NormalTok{  \}}
  \ControlFlowTok{else} \ControlFlowTok{if}\NormalTok{(x }\SpecialCharTok{==} \StringTok{"midwest"}\NormalTok{)\{}
    \FunctionTok{return}\NormalTok{(midwest\_total\_population)}
\NormalTok{  \}}
  \ControlFlowTok{else} \ControlFlowTok{if}\NormalTok{(x }\SpecialCharTok{==} \StringTok{"westcoast"}\NormalTok{)\{}
    \FunctionTok{return}\NormalTok{(west\_total\_population)}
\NormalTok{  \}}
  \ControlFlowTok{else} \ControlFlowTok{if}\NormalTok{(x }\SpecialCharTok{==} \StringTok{"south"}\NormalTok{)\{}
    \FunctionTok{return}\NormalTok{(south\_total\_population)}
\NormalTok{  \}}
  \ControlFlowTok{else}\NormalTok{\{}
    \FunctionTok{return}\NormalTok{(other\_total\_population)}
\NormalTok{  \}}
\NormalTok{\}}


\NormalTok{breaches}\SpecialCharTok{$}\NormalTok{normalized\_indiv\_affected }\OtherTok{\textless{}{-}} \FunctionTok{sapply}\NormalTok{(breaches}\SpecialCharTok{$}\NormalTok{region, normal\_function)}


\NormalTok{breaches }\OtherTok{\textless{}{-}}\NormalTok{ breaches }\SpecialCharTok{\%\textgreater{}\%}
  \FunctionTok{mutate}\NormalTok{(}\AttributeTok{normalized\_indiv\_affected =}\NormalTok{ Individuals\_Affected }\SpecialCharTok{/}\NormalTok{ normalized\_indiv\_affected)}
\end{Highlighting}
\end{Shaded}

\begin{Shaded}
\begin{Highlighting}[]
\NormalTok{breaches }\SpecialCharTok{\%\textgreater{}\%}
  \FunctionTok{ggplot}\NormalTok{(}\FunctionTok{aes}\NormalTok{(}\AttributeTok{x =}\NormalTok{ region, }\AttributeTok{y =}\NormalTok{ normalized\_indiv\_affected, }\AttributeTok{fill =}\NormalTok{ region)) }\SpecialCharTok{+}
  \FunctionTok{geom\_col}\NormalTok{() }\SpecialCharTok{+}
  \FunctionTok{ylab}\NormalTok{(}\StringTok{"Proportion of Region Affected by Breaches"}\NormalTok{)}
\end{Highlighting}
\end{Shaded}

\includegraphics{CourtneyKennedy_files/figure-latex/unnamed-chunk-38-1.pdf}

\begin{Shaded}
\begin{Highlighting}[]
\NormalTok{breaches }\SpecialCharTok{\%\textgreater{}\%}
  \FunctionTok{filter}\NormalTok{(large\_affected }\SpecialCharTok{==} \ConstantTok{TRUE}\NormalTok{) }\SpecialCharTok{\%\textgreater{}\%}
  \FunctionTok{ggplot}\NormalTok{(}\FunctionTok{aes}\NormalTok{(region, normalized\_indiv\_affected)) }\SpecialCharTok{+}
  \FunctionTok{geom\_boxplot}\NormalTok{() }\SpecialCharTok{+}
  \FunctionTok{labs}\NormalTok{(}\AttributeTok{title =} \StringTok{"Large breach {-}all"}\NormalTok{)}\SpecialCharTok{+}
  \FunctionTok{ylab}\NormalTok{(}\StringTok{"Proportion of Region Affected by Breaches"}\NormalTok{)}
\end{Highlighting}
\end{Shaded}

\includegraphics{CourtneyKennedy_files/figure-latex/unnamed-chunk-39-1.pdf}

\begin{Shaded}
\begin{Highlighting}[]
\NormalTok{breaches }\SpecialCharTok{\%\textgreater{}\%}
  \FunctionTok{filter}\NormalTok{(large\_affected }\SpecialCharTok{==} \ConstantTok{FALSE}\NormalTok{) }\SpecialCharTok{\%\textgreater{}\%}
  \FunctionTok{ggplot}\NormalTok{(}\FunctionTok{aes}\NormalTok{(region, normalized\_indiv\_affected)) }\SpecialCharTok{+}
  \FunctionTok{geom\_boxplot}\NormalTok{()}\SpecialCharTok{+}
  \FunctionTok{labs}\NormalTok{(}\AttributeTok{title =} \StringTok{"Small breach {-}all"}\NormalTok{)}\SpecialCharTok{+}
  \FunctionTok{ylab}\NormalTok{(}\StringTok{"Proportion of Region Affected by Breaches"}\NormalTok{)}
\end{Highlighting}
\end{Shaded}

\includegraphics{CourtneyKennedy_files/figure-latex/unnamed-chunk-39-2.pdf}

\begin{Shaded}
\begin{Highlighting}[]
\NormalTok{breaches }\SpecialCharTok{\%\textgreater{}\%}
  \FunctionTok{filter}\NormalTok{(large\_affected }\SpecialCharTok{==} \ConstantTok{TRUE}\NormalTok{, region }\SpecialCharTok{!=} \StringTok{"other"}\NormalTok{) }\SpecialCharTok{\%\textgreater{}\%}
  \FunctionTok{ggplot}\NormalTok{(}\FunctionTok{aes}\NormalTok{(region, normalized\_indiv\_affected)) }\SpecialCharTok{+}
  \FunctionTok{geom\_boxplot}\NormalTok{()}\SpecialCharTok{+}
  \FunctionTok{labs}\NormalTok{(}\AttributeTok{title =} \StringTok{"Large breach {-}no other"}\NormalTok{)}\SpecialCharTok{+}
  \FunctionTok{ylab}\NormalTok{(}\StringTok{"Proportion of Region Affected by Breaches"}\NormalTok{)}
\end{Highlighting}
\end{Shaded}

\includegraphics{CourtneyKennedy_files/figure-latex/unnamed-chunk-39-3.pdf}

\begin{Shaded}
\begin{Highlighting}[]
\NormalTok{breaches }\SpecialCharTok{\%\textgreater{}\%}
  \FunctionTok{filter}\NormalTok{(large\_affected }\SpecialCharTok{==} \ConstantTok{FALSE}\NormalTok{, region }\SpecialCharTok{!=} \StringTok{"other"}\NormalTok{) }\SpecialCharTok{\%\textgreater{}\%}
  \FunctionTok{ggplot}\NormalTok{(}\FunctionTok{aes}\NormalTok{(region, normalized\_indiv\_affected)) }\SpecialCharTok{+}
  \FunctionTok{geom\_boxplot}\NormalTok{()}\SpecialCharTok{+}
  \FunctionTok{labs}\NormalTok{(}\AttributeTok{title =} \StringTok{"Small breach {-}no other"}\NormalTok{)}\SpecialCharTok{+}
  \FunctionTok{ylab}\NormalTok{(}\StringTok{"Proportion of Region Affected by Breaches"}\NormalTok{)}
\end{Highlighting}
\end{Shaded}

\includegraphics{CourtneyKennedy_files/figure-latex/unnamed-chunk-39-4.pdf}
By normalizing the individuals by state population and then grouping
into regions we can compare the affect of each breach by region. The
other region, which includes PR and HI, has a much higher percentage of
their population being affected by breaches, seen in the bar chart and
the boxplots. To observe the distribution of the other regions of the US
better I removed the other region, but overall there were not any strong
trends in either the small breach or the large breach. In small
breaches, the northeast has a slightly higher distribution of percent of
individuals affected by breaches, but nothing significant engough to
make a claim about.

\textbf{- Discuss how the observed patterns support/reject your
hypotheses or answer your questions.} The state of the breach does
affect the number of individuals affected by the breach. The states with
the most individuals affected have a large city associated with them,
Virginia(Virginia Beach, Arlington), CA (Los Angles, San Francisco), IL
(Chicago), FL(Miami and Tampa), NY(New York City), TN(Nashville),
TX(Houston, San Antonio, Dallas, Austin), AL(Birmingham), MA(Boston),
NJ(Newark), UT(Salt Lake City). PR which is a territory breaks this
trend. Since most of the breaches were in the medical field, states with
large cities have larger populations and therefore have more opportunity
to affect more individuals. Also looking at the distribution of
normalized values for individuals affected, the states VA, PR, and IL
stand out, as well as the other region, which includes PR and HI, for
higher percentages of their population being affected by breaches. From
this analysis we can answer the question that if you are in a certain
state if you are more likely to be affected by a breach, by saying if
you are in VA, IL, HI, and especially PR you are more likely to be
effected by a breach.

\hypertarget{model-building}{%
\subsection{Model Building}\label{model-building}}

Load modelr

\begin{Shaded}
\begin{Highlighting}[]
\FunctionTok{library}\NormalTok{(modelr)}
\FunctionTok{options}\NormalTok{(}\AttributeTok{na.action =}\NormalTok{ na.warn)}
\end{Highlighting}
\end{Shaded}

Look at distribution of small breaches by region.

\begin{Shaded}
\begin{Highlighting}[]
\NormalTok{breaches }\SpecialCharTok{\%\textgreater{}\%}
  \FunctionTok{filter}\NormalTok{(large\_affected }\SpecialCharTok{==} \ConstantTok{FALSE}\NormalTok{) }\SpecialCharTok{\%\textgreater{}\%}
  \FunctionTok{ggplot}\NormalTok{(}\FunctionTok{aes}\NormalTok{(region, Individuals\_Affected)) }\SpecialCharTok{+}
  \FunctionTok{geom\_boxplot}\NormalTok{()}
\end{Highlighting}
\end{Shaded}

\includegraphics{CourtneyKennedy_files/figure-latex/unnamed-chunk-41-1.pdf}

Fit the model and display its predictions overlaid on the orginial data

\begin{Shaded}
\begin{Highlighting}[]
\NormalTok{small\_breaches }\OtherTok{\textless{}{-}}\NormalTok{ breaches }\SpecialCharTok{\%\textgreater{}\%}
  \FunctionTok{filter}\NormalTok{(large\_affected }\SpecialCharTok{==} \ConstantTok{FALSE}\NormalTok{)}

\NormalTok{mod }\OtherTok{\textless{}{-}} \FunctionTok{lm}\NormalTok{(Individuals\_Affected }\SpecialCharTok{\textasciitilde{}}\NormalTok{ region, }\AttributeTok{data =}\NormalTok{ small\_breaches)}

\FunctionTok{summary}\NormalTok{(mod)}
\end{Highlighting}
\end{Shaded}

\begin{verbatim}
## 
## Call:
## lm(formula = Individuals_Affected ~ region, data = small_breaches)
## 
## Residuals:
##    Min     1Q Median     3Q    Max 
##  -5349  -2588  -1544   1004  16018 
## 
## Coefficients:
##                 Estimate Std. Error t value Pr(>|t|)    
## (Intercept)       3619.9      292.2  12.389  < 2e-16 ***
## regionmidwest     -416.0      393.2  -1.058  0.29036    
## regionsouth       -120.9      365.0  -0.331  0.74062    
## regionwestcoast    173.1      401.8   0.431  0.66665    
## regionother       2333.8      885.4   2.636  0.00853 ** 
## ---
## Signif. codes:  0 '***' 0.001 '**' 0.01 '*' 0.05 '.' 0.1 ' ' 1
## 
## Residual standard error: 3920 on 942 degrees of freedom
## Multiple R-squared:  0.01139,    Adjusted R-squared:  0.007197 
## F-statistic: 2.714 on 4 and 942 DF,  p-value: 0.02882
\end{verbatim}

\begin{Shaded}
\begin{Highlighting}[]
\NormalTok{grid }\OtherTok{\textless{}{-}}\NormalTok{ small\_breaches }\SpecialCharTok{\%\textgreater{}\%}
  \FunctionTok{data\_grid}\NormalTok{(region) }\SpecialCharTok{\%\textgreater{}\%}
  \FunctionTok{add\_predictions}\NormalTok{(mod, }\StringTok{"Individuals\_Affected"}\NormalTok{)}

\FunctionTok{ggplot}\NormalTok{(small\_breaches, }\FunctionTok{aes}\NormalTok{(region, Individuals\_Affected)) }\SpecialCharTok{+}
  \FunctionTok{geom\_boxplot}\NormalTok{() }\SpecialCharTok{+}
  \FunctionTok{geom\_point}\NormalTok{(}\AttributeTok{data =}\NormalTok{ grid, }\AttributeTok{colour =} \StringTok{"red"}\NormalTok{, }\AttributeTok{size =} \DecValTok{4}\NormalTok{)}\SpecialCharTok{+}
  \FunctionTok{labs}\NormalTok{(}\StringTok{"Model for small data breaches"}\NormalTok{)}
\end{Highlighting}
\end{Shaded}

\includegraphics{CourtneyKennedy_files/figure-latex/unnamed-chunk-42-1.pdf}

\begin{Shaded}
\begin{Highlighting}[]
\NormalTok{large\_breaches }\OtherTok{\textless{}{-}}\NormalTok{ breaches }\SpecialCharTok{\%\textgreater{}\%}
  \FunctionTok{filter}\NormalTok{(large\_affected }\SpecialCharTok{==} \ConstantTok{TRUE}\NormalTok{)}

\NormalTok{mod }\OtherTok{\textless{}{-}} \FunctionTok{lm}\NormalTok{(Individuals\_Affected }\SpecialCharTok{\textasciitilde{}}\NormalTok{ region, }\AttributeTok{data =}\NormalTok{ large\_breaches)}

\FunctionTok{summary}\NormalTok{(mod)}
\end{Highlighting}
\end{Shaded}

\begin{verbatim}
## 
## Call:
## lm(formula = Individuals_Affected ~ region, data = large_breaches)
## 
## Residuals:
##     Min      1Q  Median      3Q     Max 
## -332691 -248168 -189683  -79784 4546565 
## 
## Coefficients:
##                 Estimate Std. Error t value Pr(>|t|)
## (Intercept)       220862     148313   1.489    0.139
## regionmidwest      57336     218311   0.263    0.793
## regionsouth       132573     188634   0.703    0.484
## regionwestcoast     3416     201181   0.017    0.986
## regionother       -93484     261132  -0.358    0.721
## 
## Residual standard error: 679700 on 103 degrees of freedom
## Multiple R-squared:  0.01124,    Adjusted R-squared:  -0.02716 
## F-statistic: 0.2927 on 4 and 103 DF,  p-value: 0.8822
\end{verbatim}

\begin{Shaded}
\begin{Highlighting}[]
\NormalTok{grid }\OtherTok{\textless{}{-}}\NormalTok{ large\_breaches }\SpecialCharTok{\%\textgreater{}\%}
  \FunctionTok{data\_grid}\NormalTok{(region) }\SpecialCharTok{\%\textgreater{}\%}
  \FunctionTok{add\_predictions}\NormalTok{(mod, }\StringTok{"Individuals\_Affected"}\NormalTok{)}

\FunctionTok{ggplot}\NormalTok{(large\_breaches, }\FunctionTok{aes}\NormalTok{(region, Individuals\_Affected)) }\SpecialCharTok{+}
  \FunctionTok{geom\_boxplot}\NormalTok{() }\SpecialCharTok{+}
  \FunctionTok{geom\_point}\NormalTok{(}\AttributeTok{data =}\NormalTok{ grid, }\AttributeTok{colour =} \StringTok{"red"}\NormalTok{, }\AttributeTok{size =} \DecValTok{4}\NormalTok{) }\SpecialCharTok{+}
  \FunctionTok{labs}\NormalTok{(}\StringTok{"Model for large data breaches"}\NormalTok{)}
\end{Highlighting}
\end{Shaded}

\includegraphics{CourtneyKennedy_files/figure-latex/unnamed-chunk-43-1.pdf}

\begin{Shaded}
\begin{Highlighting}[]
\NormalTok{large\_breaches }\OtherTok{\textless{}{-}}\NormalTok{ breaches }\SpecialCharTok{\%\textgreater{}\%}
  \FunctionTok{filter}\NormalTok{(large\_affected }\SpecialCharTok{==} \ConstantTok{TRUE}\NormalTok{)}

\NormalTok{mod }\OtherTok{\textless{}{-}} \FunctionTok{lm}\NormalTok{(normalized\_indiv\_affected }\SpecialCharTok{\textasciitilde{}}\NormalTok{ region, }\AttributeTok{data =}\NormalTok{ large\_breaches)}

\FunctionTok{summary}\NormalTok{(mod)}
\end{Highlighting}
\end{Shaded}

\begin{verbatim}
## 
## Call:
## lm(formula = normalized_indiv_affected ~ region, data = large_breaches)
## 
## Residuals:
##       Min        1Q    Median        3Q       Max 
## -0.021466 -0.003319 -0.002471 -0.001429  0.072436 
## 
## Coefficients:
##                   Estimate Std. Error t value Pr(>|t|)    
## (Intercept)      0.0039398  0.0028369   1.389    0.168    
## regionmidwest    0.0001437  0.0041758   0.034    0.973    
## regionsouth     -0.0010790  0.0036081  -0.299    0.766    
## regionwestcoast -0.0010368  0.0038481  -0.269    0.788    
## regionother      0.0226028  0.0049948   4.525 1.62e-05 ***
## ---
## Signif. codes:  0 '***' 0.001 '**' 0.01 '*' 0.05 '.' 0.1 ' ' 1
## 
## Residual standard error: 0.013 on 103 degrees of freedom
## Multiple R-squared:  0.2204, Adjusted R-squared:  0.1901 
## F-statistic: 7.279 on 4 and 103 DF,  p-value: 3.34e-05
\end{verbatim}

\begin{Shaded}
\begin{Highlighting}[]
\NormalTok{grid }\OtherTok{\textless{}{-}}\NormalTok{ large\_breaches }\SpecialCharTok{\%\textgreater{}\%}
  \FunctionTok{data\_grid}\NormalTok{(region) }\SpecialCharTok{\%\textgreater{}\%}
  \FunctionTok{add\_predictions}\NormalTok{(mod, }\StringTok{"normalized\_indiv\_affected"}\NormalTok{)}

\FunctionTok{ggplot}\NormalTok{(large\_breaches, }\FunctionTok{aes}\NormalTok{(region, normalized\_indiv\_affected)) }\SpecialCharTok{+}
  \FunctionTok{geom\_boxplot}\NormalTok{() }\SpecialCharTok{+}
  \FunctionTok{geom\_point}\NormalTok{(}\AttributeTok{data =}\NormalTok{ grid, }\AttributeTok{colour =} \StringTok{"red"}\NormalTok{, }\AttributeTok{size =} \DecValTok{4}\NormalTok{)}\SpecialCharTok{+}
  \FunctionTok{labs}\NormalTok{(}\StringTok{"Model for large data breaches with normalized indiv affected"}\NormalTok{)}
\end{Highlighting}
\end{Shaded}

\includegraphics{CourtneyKennedy_files/figure-latex/unnamed-chunk-44-1.pdf}

\begin{Shaded}
\begin{Highlighting}[]
\NormalTok{small\_breaches }\OtherTok{\textless{}{-}}\NormalTok{ breaches }\SpecialCharTok{\%\textgreater{}\%}
  \FunctionTok{filter}\NormalTok{(large\_affected }\SpecialCharTok{==} \ConstantTok{FALSE}\NormalTok{)}

\NormalTok{mod }\OtherTok{\textless{}{-}} \FunctionTok{lm}\NormalTok{(normalized\_indiv\_affected }\SpecialCharTok{\textasciitilde{}}\NormalTok{ region, }\AttributeTok{data =}\NormalTok{ small\_breaches)}

\FunctionTok{summary}\NormalTok{(mod)}
\end{Highlighting}
\end{Shaded}

\begin{verbatim}
## 
## Call:
## lm(formula = normalized_indiv_affected ~ region, data = small_breaches)
## 
## Residuals:
##        Min         1Q     Median         3Q        Max 
## -1.115e-03 -3.232e-05 -1.769e-05  1.168e-05  2.463e-03 
## 
## Coefficients:
##                   Estimate Std. Error t value Pr(>|t|)    
## (Intercept)      6.457e-05  1.089e-05   5.930 4.24e-09 ***
## regionmidwest   -1.754e-05  1.465e-05  -1.197  0.23147    
## regionsouth     -3.625e-05  1.360e-05  -2.665  0.00783 ** 
## regionwestcoast -1.548e-05  1.497e-05  -1.034  0.30159    
## regionother      1.176e-03  3.299e-05  35.644  < 2e-16 ***
## ---
## Signif. codes:  0 '***' 0.001 '**' 0.01 '*' 0.05 '.' 0.1 ' ' 1
## 
## Residual standard error: 0.0001461 on 942 degrees of freedom
## Multiple R-squared:  0.6059, Adjusted R-squared:  0.6043 
## F-statistic: 362.1 on 4 and 942 DF,  p-value: < 2.2e-16
\end{verbatim}

\begin{Shaded}
\begin{Highlighting}[]
\NormalTok{grid }\OtherTok{\textless{}{-}}\NormalTok{ small\_breaches }\SpecialCharTok{\%\textgreater{}\%}
  \FunctionTok{data\_grid}\NormalTok{(region) }\SpecialCharTok{\%\textgreater{}\%}
  \FunctionTok{add\_predictions}\NormalTok{(mod, }\StringTok{"normalized\_indiv\_affected"}\NormalTok{)}

\FunctionTok{ggplot}\NormalTok{(small\_breaches, }\FunctionTok{aes}\NormalTok{(region, normalized\_indiv\_affected)) }\SpecialCharTok{+}
  \FunctionTok{geom\_boxplot}\NormalTok{() }\SpecialCharTok{+}
  \FunctionTok{geom\_point}\NormalTok{(}\AttributeTok{data =}\NormalTok{ grid, }\AttributeTok{colour =} \StringTok{"red"}\NormalTok{, }\AttributeTok{size =} \DecValTok{4}\NormalTok{)}\SpecialCharTok{+}
  \FunctionTok{labs}\NormalTok{(}\StringTok{"Model for small data breaches with normalized indiv affected"}\NormalTok{)}
\end{Highlighting}
\end{Shaded}

\includegraphics{CourtneyKennedy_files/figure-latex/unnamed-chunk-45-1.pdf}

Overall the models do not do a good job of predicting the number of
individuals affected by a breach, or the percentage of individuals
affected out of a state population. This is seen by all of the models
only having a significant p value (represented by the 2 or 3 stars) for
the other region. The model that has the strongest significance is the
normalized individuals affected and small data breaches by region, which
has a 2 p values that are significant, for the southern region and other
region. Since most of the p values are not statistically significant the
models can not be accurately used to predict the individuals affected by
region.

\end{document}
